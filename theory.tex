%!TEX root =  index.tex

\section{Theory}


% TODO: wofür brauchen wir das?
\begin{lemma}
Let $f \colon \M_2(\C) \rightarrow \C$ be a Hermitian Modular form of weight $k$. Let $S \in \PM_2(\C)$.
Then, $f(S \tau) \colon \HalfPlane_1 \subseteq \C \rightarrow \C$ is an elliptic modular form of weight $2k$ to $\Gamma_0(l)$, where $l$ is the denominator of $S^{-1}$.  %for some matrix $S \in \M_2(\Z)$ with $\Gamma(S) \subseteq \SL_2(\Z)$.
\end{lemma}


% TODO: wofür brauchen wir das?
% von Notizen im Block
% Prop 7.3 von Poor für herm Modulformen
\begin{lemma}
Prop 7.3. von Poor für herm Modulformen.
$\Gamma(\mathcal{L}) \supseteq \Gamma_0(l)$ for $l \in \Z^+, ls^{-1} \in \mathcal{P}_n(\curlO)$.
% \Gamma ist kein Gitter, sondern eine diskrete Untergruppe
% Beweisskizze in Unterlagen
% L symmetrische, ganze Matrix => \Gamma(L) \subseteq \SL{2}(\ZZ)
% \curlL ist ein polarisiertes Gitter, welches wir zur Vereinfachung als gewöhliche symmetrische, ganz Matrix ansehen
\end{lemma}


We want to calculate a generating set for the Fourier expansions of Hermitian modular forms Now we will formulate the main algorithm of our work.

\begin{algo}
We have the Hermitian modular form degree $n = 2$ fixed, as well as some $\Delta$ (for now, $\Delta \in \Set{3,4,8}$). Then we select some form weight $k \in \Z$ ($k \in \Set{1,\dots,20}$ or so), some $\curlO \subseteq \Q(\sqrt{-\Delta})$ and some subgroup $\Gamma$ of $\Sp_2(\curlO)$. Then we select an abel character $\nu \colon \Gamma \rightarrow \C^\times$ of $\Sp_2(\curlO)$.

We define the index set
\[ \Lambda := \Set{0 \le \SimpleMatrix{a}{b}{\overline b}{c} \in \M_2(\curlO^\#)}{a,c \in \Z } . \]

We start with $l = 1$ and increase it but only use the square-free numbers.

Fix $B \in \N$ as a limit. Select a precision
\[ \F := \Set{\SimpleMatrix{a}{b}{\overline b}{c}}{0 \le a , c < B, b \in \curlO^\#} \subseteq \Lambda . \]
% \Lambda sind die die Indizes der Fourier-Entwicklung einer Hermitischen Modulform.

\begin{enumerate}
\item Set $\mathcal{S} = \{\}$,
\item Enumerate matrices $S \in \M_2^T(\Z)$, and set $\mathcal{S} \leftarrow \mathcal{S} \cup \{ S \}$ and for each time you add a new matrix perform the following steps.

%Select a set of matrices $\mathcal{S} \subseteq \M_2^T(\Z)$ with $0 < S \in \mathcal{S}$.
%Make $\mathcal{S}$ big enough.
%Now, for some $S \in \mathcal{S}$:

\item
\[ \mathcal{M}_{k,\mathcal{S},\F}^H = \Set{ (f [S])_{S\in\mathcal{S}} }{f \in \Q^\F \text{is $\GL_2(\curlO)$ invariant}} \subseteq \bigoplus_S \Q^{\F(S)} , \]
% unter GL_2(\curlO) invariant:
% also Fourier-Entwicklung unter der Operation von \GL{2}(\cO) invariant:
% also wenn a(T), T \in \Lambda die Fourier-Koeffizienten bezeichnen, dann gilt \det(U)^k a(T[U]) = a(T) für alle U \in \GL{2}(\cO)
\[ \mathcal{M}_{k, \mathcal{S}} = \bigoplus_S \FE_{\F(S)}(\M_k(\Gamma(l_S))) \]

\item
If
\[ \dim \mathcal{M}_{k,\mathcal{S},\F}^H \cap \mathcal{M}_{k, \mathcal{S}}
= \dim M^H_k , \]
then we are ready and we can reconstruct the Fourier expansion in the following way: ...

If not, then return to Step 2, and enlarge $\mathcal{S}$.
\end{enumerate}
\end{algo}

%Matrix-repr of f \in (\Q^{\CurlfF[G]})^G_Chi \mapsto (\sum_{tr(ST)=n} a(T))_{n,S}
%( )^{G, chi} meine ich die Menge der Elemente von Q^\cF[G], die a(T[g]) = chi(g) a(T) erfüllen
% Eine Basis davon kannst du dir mittels der Reduktionen von T überlegen
