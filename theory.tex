%!TEX root =  index.tex

\section{Theory}


% TODO: wofür brauchen wir das?
\begin{lemma}
Let $f \colon \M_2(\C) \rightarrow \C$ be a Hermitian Modular form of weight $k$.
Then, $f(S \tau) \colon \HalfPlane_1 \subseteq \C \rightarrow \C$ is an eliptic modular form of weight $2k$ for some matrix $S \in \M_2(\Z)$ with $\Gamma(S) \subseteq \SL_2(\Z)$.
\end{lemma}


% TODO: wofür brauchen wir das?
% von Notizen im Block
% Prop 7.3 von Poor für herm Modulformen
\begin{lemma}
Prop 7.3. von Poor für herm Modulformen.
$\Gamma(\mathcal{L}) \supseteq \Gamma_0(l)$ for $l \in \Z^+, ls^{-1} \in \mathcal{P}_n(\curlO)$.
% \Gamma ist kein Gitter, sondern eine diskrete Untergruppe
% Beweisskizze in Unterlagen
% L symmetrische, ganze Matrix => \Gamma(L) \subseteq \SL{2}(\ZZ)
% \curlL ist ein polarisiertes Gitter, welches wir zur Vereinfachung als gewöhliche symmetrische, ganz Matrix ansehen
\end{lemma}


Now we will formulate the main algorithm of our work.

\begin{algo}
\begin{enumerate}
\item Select a set of matrices $\mathcal{S} \subseteq \M_2^T(\Z)$ with $0 < S \in \mathcal{S}$. Make $\mathcal{S}$ big enough. Now, for some $S \in \mathcal{S}$:

\item Fix $B \in \N$ as a limit. Or select a precision
\[ \F = \Set{\SimpleMatrix{a}{b}{\overline b}{c}}{0 \le a c < B} \subseteq \Lambda , \]
where
\[ \Lambda := \Set{0 \le \SimpleMatrix{a}{b}{\overline b}{c} \in \M_2(\curlO^\#)}{a,c \in \Z } . \]
% \Lambda sind die die Indizes der Fourier-Entwicklung einer Hermitischen Modulform.

\item \[ \mathcal{M}_{k,\mathcal{S},\F}^H = \Set{f [S]}{f \in \Q^\F} , \]
\[ \mathcal{M}_{k,S} = \FE_{\F(S)}(\M_k(\Gamma(S))) \]

\item If
\[ \dim \mathcal{M}_{k,\mathcal{S},\F}^H \cap \bigoplus_{S \in \mathcal{S}} \mathcal{M}_{k,S}
= \dim M^H_k , \]
then we are ready and we can reconstruct the Fourier expansion in the following way: ...

\end{enumerate}
\end{algo}

%Matrix-repr of f \in (\Q^{\CurlfF[G]})^G_Chi \mapsto (\sum_{tr(ST)=n} a(T))_{n,S}
%( )^{G, chi} meine ich die Menge der Elemente von Q^\cF[G], die a(T[g]) = chi(g) a(T) erfüllen
% Eine Basis davon kannst du dir mittels der Reduktionen von T überlegen
