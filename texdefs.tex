%!TEX root =  index.tex

% some useful stuff:
% http://www.automata.rwth-aachen.de/material/skripte/latex/latex.pdf
% http://en.wikibooks.org/wiki/LaTeX/Mathematics
% http://en.wikibooks.org/wiki/LaTeX/Advanced_Mathematics
% http://en.wikibooks.org/wiki/LaTeX/Theorems

\theoremstyle{plain}\newtheorem{lemma}{Lemma}[section]
\theoremstyle{plain}\newtheorem{theorem}[lemma]{Theorem}
\theoremstyle{definition}\newtheorem{mydef}[lemma]{Definition}
\theoremstyle{definition}\newtheorem{algo}[lemma]{Algorithm}
\theoremstyle{plain}\newtheorem{example}[lemma]{Example}
\theoremstyle{definition}\newtheorem{sexample}[lemma]{Example}
\theoremstyle{definition}\newtheorem{remark}[lemma]{Remark}
\theoremstyle{definition}\newtheorem{prelim}[lemma]{Preliminaries}

% http://tex.stackexchange.com/questions/5767/how-to-get-more-complete-references
\Crefname{section}{Chapter}{Chapters}
\crefname{section}{chapter}{chapters}
\Crefname{subsection}{Section}{Sections}
\crefname{subsection}{section}{sections}

\crefname{sexample}{example}{examples}
\crefname{mydef}{definition}{definitions}

\newenvironment{simpleexample}[0]%
{\begin{sexample}}%
{\qed \end{sexample}}

\newcommand{\F}{\mathcal{F}}
\newcommand{\K}{\mathbb{K}}
\newcommand{\Z}{\mathbb{Z}}
\newcommand{\Q}{\mathbb{Q}}
\newcommand{\C}{\mathbb{C}}
\newcommand{\R}{\mathbb{R}}
\newcommand{\N}{\mathbb{N}}
\newcommand{\B}{\mathbb{B}}
\newcommand{\HalfPlane}{\mathbb{H}}
\newcommand{\SiegelHalfPlane}{\mathcal{H}}
\newcommand{\Power}{\wp} % dont use \mathcal{P} because of \PM

\newcommand{\Trans}{\operatorname{Trans}}
\newcommand{\Rot}{\operatorname{Rot}}

\newcommand{\mathtext}[1]{\textup{\textrm{#1}}}
\newcommand{\tr}{\mathtext{tr}}
%\newcommand{\invar}[2]{\Set{x \in #1}{\text{$x$ is $#2$ invariant}}}
\newcommand{\invar}[2]{{\left(#1\right)}^{#2}}
\newcommand{\invarF}[2]{{#1}^{#2}}

% got some help here: http://tex.stackexchange.com/questions/13554/define-something-like-lim-but-for-another-name

\newcommand{\M}{\operatorname{Mat}}
\newcommand{\PM}{\operatorname{\mathcal{P}}}
\newcommand{\GL}{\operatorname{GL}}
\newcommand{\SL}{\operatorname{SL}}
\newcommand{\Sp}{\operatorname{Sp}}
\newcommand{\Orth}{\operatorname{O}}
\newcommand{\Her}{\operatorname{Her}}
\newcommand{\curlO}{\mathcal{O}}
%\newcommand{\det}{\operatorname{det}}
\newcommand{\FE}{\mathcal{FE}} % Fourier expansion

\newcommand{\ModFormSpace}[3]{\operatorname{\mathcal{M}}^{#1}_#2 (#3)}
\newcommand{\ESpace}[2]{\ModFormSpace{}{#1}{#2}}
\newcommand{\SSpaceN}[3]{\ModFormSpace{\SiegelHalfPlane_{#1}}{#2}{#3}}
\newcommand{\HSpaceN}[3]{\ModFormSpace{\HalfPlane_{#1}}{#2}{#3}}
\newcommand{\SSpace}[2]{\SSpaceN{2}{#1}{#2}}
\newcommand{\HSpace}[2]{\HSpaceN{2}{#1}{#2}}

\newcommand{\SmallMatrix}[4]{\left( \begin{smallmatrix}
#1 & #2 \\
#3 & #4 \end{smallmatrix} \right)}

\newcommand{\SimpleMatrix}[4]{\left( \begin{array}{cc}
#1 & #2 \\
#3 & #4 \end{array} \right)}

\newcommand{\existsinf}{\exists^\omega}
\newcommand{\overx}{\overset{\times}}
%\newcommand{\overx}{\stackrel{\times}}
\newcommand{\Ax}{\overx{\A}}

\newcommand{\FPrecisionLimit}[1]{\F(#1)}

\newcommand{\defword}[1]{{\bf #1}}

% http://tex.stackexchange.com/questions/114997/how-do-i-define-a-custom-verbatim-command
% http://tex.stackexchange.com/questions/117979/how-to-do-newcommand-filepath1-verb1
\lstdefinestyle{inline}{
    columns=fullflexible,
    breaklines=false,
    basicstyle=\itshape
}
\newcommand{\ifuncname}[1]{\lstinline[style=inline]!#1!}
%\newcommand{\ifuncname}[1]{$#1$}
%\newcommand{\ifilename}[1]{\verb!#1!}
%\DeclareUrlCommand\ifuncname{}
\DeclareUrlCommand\ifilename{}

% inspired by http://ftp.fernuni-hagen.de/ftp-dir/pub/mirrors/www.ctan.org/macros/latex/contrib/braket/braket.sty
\def\mid@vertical{\mskip1mu\vrule\mskip1mu}
\def\midvert{\egroup\;\mid@vertical\;\bgroup}
\NewDocumentCommand\Set{mg}{%
    \IfNoValueTF{#2}{%
        \ensuremath{\left\{ #1 \right\}}%
    }{%
        \ensuremath{\left\{ {#1} \;\mid@vertical\; {#2} \right\}}%
    }%
}

%\newcommand{\SetS}[1]{\bigl\{ #1 \bigr\}}
%\newcommand{\SetC}[2]{\bigl\{ #1 \bigm| #2 \bigr\}}
%\DeclarePairedDelimiterX\SetC[2]{\lbrace}{\rbrace}{ #1 \,\delimsize|\, #2 }

%\newcommand{\ab	s}[1]{\mathopen| #1 \mathclose|}
%\newcommand{\Abs}[1]{\left| #1 \right|}
\newcommand{\abs}[1]{\left| #1 \right|}

% http://de.wikibooks.org/wiki/LaTeX-W%C3%B6rterbuch:_today
\def\monthgerman{\ifcase\month \or
  Januar\or Februar\or M\"arz\or April\or Mai\or Juni\or
  Juli\or August\or September\or Oktober\or November\or Dezember\fi}
\def\todaygerman{\number\day.~\monthgerman\space\number\year}


% new page for every section
\let\stdsection\section
\renewcommand\section{\newpage\stdsection}
