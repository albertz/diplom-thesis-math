\documentclass[twoside,openright]{article}
%%!TEX encoding = UTF-8 Unicode
\usepackage[utf8]{inputenc}
\usepackage{palatino} %Schriftart
%\usepackage{ngerman}
%\usepackage[ps2pdf,a4paper,colorlinks]{hyperref}
%\usepackage[a4paper,colorlinks]{hyperref}
\usepackage[a4paper]{hyperref}
\usepackage[a4paper,%
	inner=3.5cm,%
	outer=3.5cm,%
	top=4cm,%
	bottom=4cm,%
	marginparwidth=2.5cm,%
	marginparsep=0.3cm,%
	includehead]{geometry}
\usepackage{makeidx}
\usepackage[nottoc,numbib]{tocbibind}
\usepackage{titlesec}
\usepackage[fleqn]{amsmath}
\usepackage{amsthm}
\usepackage{amstext}
\usepackage{amssymb}
\usepackage{mathtools}
\usepackage{xparse}
\usepackage{url}
\usepackage{cleveref}
\usepackage{mdframed}
\usepackage{tikz}
\usetikzlibrary{automata,positioning}
% doesnt work?
%\usepackage{vaucanson-g}
% evtl auch gastex. siehe: http://www.automata.rwth-aachen.de/material/skripte/latex/latex.pdf

\pagestyle{headings}

\hypersetup{%
	pdftitle = {Hermitian Modular Forms for Fields of Low Discriminant},%
	pdfsubject = {},%
	pdfauthor = {Albert Zeyer}%
}

% funktioniert nicht?
%\makeidx

%!TEX root =  index.tex

% some useful stuff:
% http://www.automata.rwth-aachen.de/material/skripte/latex/latex.pdf
% http://en.wikibooks.org/wiki/LaTeX/Mathematics
% http://en.wikibooks.org/wiki/LaTeX/Advanced_Mathematics
% http://en.wikibooks.org/wiki/LaTeX/Theorems

\theoremstyle{plain}\newtheorem{lemma}{Lemma}[section]
\theoremstyle{plain}\newtheorem{theorem}[lemma]{Theorem}
\theoremstyle{definition}\newtheorem{mydef}[lemma]{Definition}
\theoremstyle{definition}\newtheorem{algo}[lemma]{Algorithm}
\theoremstyle{plain}\newtheorem{example}[lemma]{Example}
\theoremstyle{definition}\newtheorem{sexample}[lemma]{Example}
\theoremstyle{definition}\newtheorem{remark}[lemma]{Remark}
\theoremstyle{definition}\newtheorem{prelim}[lemma]{Preliminaries}

% http://tex.stackexchange.com/questions/5767/how-to-get-more-complete-references
\Crefname{section}{Chapter}{Chapters}
\crefname{section}{chapter}{chapters}
\Crefname{subsection}{Section}{Sections}
\crefname{subsection}{section}{sections}

\crefname{sexample}{example}{examples}
\crefname{mydef}{definition}{definitions}

\newenvironment{simpleexample}[0]%
{\begin{sexample}}%
{\qed \end{sexample}}

\newcommand{\F}{\mathcal{F}}
\newcommand{\K}{\mathbb{K}}
\newcommand{\Z}{\mathbb{Z}}
\newcommand{\Q}{\mathbb{Q}}
\newcommand{\C}{\mathbb{C}}
\newcommand{\R}{\mathbb{R}}
\newcommand{\N}{\mathbb{N}}
\newcommand{\B}{\mathbb{B}}
\newcommand{\HalfPlane}{\mathbb{H}}
\newcommand{\SiegelHalfPlane}{\mathcal{H}}
\newcommand{\Power}{\wp} % dont use \mathcal{P} because of \PM

\newcommand{\Trans}{\operatorname{Trans}}
\newcommand{\Rot}{\operatorname{Rot}}

\newcommand{\mathtext}[1]{\textup{\textrm{#1}}}
\newcommand{\tr}{\mathtext{tr}}
%\newcommand{\invar}[2]{\Set{x \in #1}{\text{$x$ is $#2$ invariant}}}
\newcommand{\invar}[2]{{\left(#1\right)}^{#2}}
\newcommand{\invarF}[2]{{#1}^{#2}}

% got some help here: http://tex.stackexchange.com/questions/13554/define-something-like-lim-but-for-another-name

\newcommand{\M}{\operatorname{Mat}}
\newcommand{\PM}{\operatorname{\mathcal{P}}}
\newcommand{\GL}{\operatorname{GL}}
\newcommand{\SL}{\operatorname{SL}}
\newcommand{\Sp}{\operatorname{Sp}}
\newcommand{\Orth}{\operatorname{O}}
\newcommand{\Her}{\operatorname{Her}}
\newcommand{\curlO}{\mathcal{O}}
%\newcommand{\det}{\operatorname{det}}
\newcommand{\FE}{\mathcal{FE}} % Fourier expansion

\newcommand{\ModFormSpace}[3]{\operatorname{\mathcal{M}}^{#1}_#2 (#3)}
\newcommand{\ESpace}[2]{\ModFormSpace{}{#1}{#2}}
\newcommand{\SSpaceN}[3]{\ModFormSpace{\SiegelHalfPlane_{#1}}{#2}{#3}}
\newcommand{\HSpaceN}[3]{\ModFormSpace{\HalfPlane_{#1}}{#2}{#3}}
\newcommand{\SSpace}[2]{\SSpaceN{2}{#1}{#2}}
\newcommand{\HSpace}[2]{\HSpaceN{2}{#1}{#2}}

\newcommand{\SmallMatrix}[4]{\left( \begin{smallmatrix}
#1 & #2 \\
#3 & #4 \end{smallmatrix} \right)}

\newcommand{\SimpleMatrix}[4]{\left( \begin{array}{cc}
#1 & #2 \\
#3 & #4 \end{array} \right)}

\newcommand{\existsinf}{\exists^\omega}
\newcommand{\overx}{\overset{\times}}
%\newcommand{\overx}{\stackrel{\times}}
\newcommand{\Ax}{\overx{\A}}

\newcommand{\FPrecisionLimit}[1]{\F(#1)}

\newcommand{\defword}[1]{{\bf #1}}

% http://tex.stackexchange.com/questions/114997/how-do-i-define-a-custom-verbatim-command
% http://tex.stackexchange.com/questions/117979/how-to-do-newcommand-filepath1-verb1
\lstdefinestyle{inline}{
    columns=fullflexible,
    breaklines=false,
    basicstyle=\itshape
}
%\newcommand{\ifuncname}[1]{\lstinline[style=inline]!#1!}
\newcommand{\ifuncname}[1]{\lstinline[style=inline]{#1}}
%\newcommand{\ifuncname}[1]{$#1$}
%\newcommand{\ifilename}[1]{\verb!#1!}
%\DeclareUrlCommand\ifuncname{}
%\DeclareUrlCommand\ifilename{}
\newcommand{\ifilename}[1]{\protect\path{#1}}

% inspired by http://ftp.fernuni-hagen.de/ftp-dir/pub/mirrors/www.ctan.org/macros/latex/contrib/braket/braket.sty
\def\mid@vertical{\mskip1mu\vrule\mskip1mu}
\def\midvert{\egroup\;\mid@vertical\;\bgroup}
\NewDocumentCommand\Set{mg}{%
    \IfNoValueTF{#2}{%
        \ensuremath{\left\{ #1 \right\}}%
    }{%
        \ensuremath{\left\{ {#1} \;\mid@vertical\; {#2} \right\}}%
    }%
}

%\newcommand{\SetS}[1]{\bigl\{ #1 \bigr\}}
%\newcommand{\SetC}[2]{\bigl\{ #1 \bigm| #2 \bigr\}}
%\DeclarePairedDelimiterX\SetC[2]{\lbrace}{\rbrace}{ #1 \,\delimsize|\, #2 }

%\newcommand{\ab	s}[1]{\mathopen| #1 \mathclose|}
%\newcommand{\Abs}[1]{\left| #1 \right|}
\newcommand{\abs}[1]{\left| #1 \right|}

% http://de.wikibooks.org/wiki/LaTeX-W%C3%B6rterbuch:_today
\def\monthgerman{\ifcase\month \or
  Januar\or Februar\or M\"arz\or April\or Mai\or Juni\or
  Juli\or August\or September\or Oktober\or November\or Dezember\fi}
\def\todaygerman{\number\day.~\monthgerman\space\number\year}


% new page for every section
\let\stdsection\section
\renewcommand\section{\newpage\stdsection}


\begin{document}
\title{Hermitian Modular Forms for Fields of Low Discriminant}
\author{Albert Zeyer}
\date{\today}

% http://en.wikibooks.org/wiki/LaTeX/Title_Creation
\begin{titlepage}
\begin{center}
\setlength{\parskip}{2ex plus0.5ex minus0.2ex}
\setlength{\baselineskip}{5ex}
\textsc{\LARGE Hermitian Modular Forms for Fields of Low Discriminant}\\[2cm]
%\textsc{\large Berechnungen Hermitescher Modulformen}\\[1.5cm]

\setlength{\baselineskip}{3ex}

\textsc{Diploma Thesis} \\
in Mathematics \\[0.7cm]

by \\
Albert Zeyer \\[3cm]

submitted to the \\
Faculty of Mathematics, Computer Science and Natural Science of \\
RWTH Aachen University \\[1.5cm]

October 2012 \\
revised version from \today \\[1.5cm]

Supervisor: Prof. Dr. Aloys Krieg \\
Second examiner: Dr. Martin Raum \\[1.5cm]

written at the \\
Lehrstuhl A für Mathematik \\
%Logic and Theory of Discrete Systems \\
Prof. Dr. A. Krieg

\end{center}
\end{titlepage}


% dont need that for the revised version
%% empty page
%\newpage
%\thispagestyle{empty}
%\mbox{}
%
%\begin{titlepage}
%\setlength{\parindent}{0pt}
%\setlength{\parskip}{3ex}
%\textbf{\large Erklärung}
%
%Hiermit versichere ich, dass ich diese Arbeit selbstständig verfasst und keine anderen als die angegebenen Quellen und Hilfsmittel benutzt sowie Zitate kenntlich gemacht habe.\\
%
%Aachen, den \todaygerman
%\end{titlepage}
%
%% empty page
%\newpage
%\thispagestyle{empty}
%\mbox{}

\newpage
\thispagestyle{empty}
\setlength{\parskip}{1.2ex}
\tableofcontents
\newpage


% http://www.automata.rwth-aachen.de/material/skripte/latex/latex.pdf
\setlength{\parindent}{0pt}
\setlength{\parskip}{2ex plus0.5ex minus0.2ex}
\setlength{\baselineskip}{3ex}
\renewcommand{\baselinestretch}{1.5}

% http://www.ctex.org/documents/packages/layout/titlesec.pdf
\titleformat{\section}[display]{\LARGE\bfseries}{Chapter \thesection}{0.5em}{}{}
\titlespacing*{\section}{0pt}{0pt}{2em}

%!TEX root =  index.tex

\section{Introduction}

We develop an algorithm to compute Fourier expansions of Hermitian modular forms of degree 2 over $\Sp_2(\curlO)$ for $\curlO \subseteq \Q(\sqrt{-\Delta})$, $\Delta \in \Set{3,4,8}$.

In \cite{PoorYuen07Comp}, spaces of Siegel modular cusp forms are calculated. It uses a linear reduction of Siegel modular forms to Elliptic modular forms and gains information from there. This is very similar to what we are doing with Hermitian modular forms.

A similar algorithm was developed in \cite[Algorithm 4.3]{Raum12Jacobi} for Jacobi forms.

We are doing the same for Hermitian modular forms. We can calculate the dimension of the Hermitian modular forms vectorspace and we also can calculate the vectorspace of Elliptic modular forms. We develop a method to restrict Hermitian modular forms to Elliptic modular forms and gain information from this relation. We can thus restrict the space of possible Fourier expansions. By repeating that, we hope to reduce the dimension so far that we eventually can describe the vector space of Fourier expansion of Hermitian modular forms.

Another further method to gain information is to calculate cusp expansions of Elliptic modular form. We can also restrict the Hermitian modular forms to those cusp expansions of Elliptic modular forms and we gain information from that relation in the same way as before. It seems likely that this gives enough information to eventually reduce the dimension enough.

Along with the theoretical work, the algorithm has also been implemented. The implementation has been done with the Sage (\cite{sage}) framework. It is implemented in C++ (\cite{cpp}), Cython (\cite{cython}) and Python (\cite{python}). The code can be found on GitHub (\cite{Zeyer13Github}) and another backup might be on \cite{Zeyer13Homepage}. We haven't been able to get any results yet, though, which is left open for further work.

In \cref{chapter:prelim}, we introduce the reader to our notation. We also define all the basic concepts as well as introduce to Elliptic, Siegel and Hermitian modular forms.

In \cref{chapter:theory}, we develop and work out all the theory of the methods for the algorithm. We also describe the algorithm itself in more detail.

In \cref{chapter:impl}, we describe all the details about the implementation as well as develop further formulas needed for the implementation, such as how to iterate through specific sets of matrices and how to calculate with our set of complex numbers.

We conclude with \cref{chapter:conclusion}. We refer to further work and we analyse the state of this algorithm and its implementation.

%!TEX root =  index.tex

%\section{Background results}

\section{Preliminaries}

$\N$ denotes the set $\Set{1,2,3,\dots}$, $\N_0 = \N \cup \Set{0}$ and $\Z$ are all \defword{integers}. $\Q$ are all the \defword{rational numbers}, $\R$ are the \defword{real numbers} and $\C$ are the \defword{complex numbers}. $\R^+ := \Set{x\in\R}{x>0}$, $\R^\times$ and $\C^\times$ denotes all non-zero numbers.

Let $\M_n(R)$ be the set of all $n \times n$ \defword{matrices} over some commutative ring $R$.
Likewise, $\M_n^T(R)$ are the \defword{symmetric} $n \times n$ matrices.
$X^T$ is the \defword{transposed} matrix of $X \in \M_n(R)$.
$\overline{Z}$ is the \defword{conjugated} matrix of $Z \in \M_n(\C)$.
For $R \subseteq \C$, $\overline{R} \subseteq R$, the set of \defword{Hermitian matrices} in $R$ is defined as
\[ \Her_n(R) = \Set{Z \in \M_n(R)}{\overline{Z}^T = Z} . \]
A matrix $Y \in \M_n(\C)$ is greater $0$ if and only if $\forall x \in \C^n - \Set{0} \colon Y[x] := \overline{x}^{T} Y x \in \R^+$.
Such matrices are called the \defword{positive definitive matrices}, defined by
\[ \PM_n(R) = \Set{X \in \M_n(R)}{X > 0} \]
for $R \subseteq \C$. Note that $\PM_n(R) \subseteq \Her_n(R)$, i.e. all positive definite matrices are Hermitian. For a matrix over $\R$, it means that it is also symmetric.

For $A,X \in \M_n(\C)$, we define $A[X] := \overline{X}^T A X$. 
The \defword{denominator} of a matrix $Z \in \M_n(\Q)$ is the smallest number $x \in \N$ such that $x Z \in \M_n(\Z)$. We also write $\operatorname{denom}(Z) = x$. $1_n \in \M_n(\Z)$ denotes the \defword{identity matrix}. We use the \defword{Gauß notation} $[a,b,c] := \SmallMatrix{a}{b}{\overline b}{c} \in \Her_n(\C)$.

The \defword{general linear group} is defined by
\[ \GL_n(R) = \Set{X \in \M_n(R)}{\text{$\det(X)$ is a unit in $R$}} \]
and the \defword{special linear group} by
\[ \SL_n(R) = \Set{X \in \M_n(R)}{\det(X) = 1} . \] % Poor
The \defword{orthogonal group} is defined by
\[ \Orth_n(R) = \Set{X \in \GL_n(R)}{X^T 1_n X = 1_n} \subseteq \GL_n(R) . \] %Poor
The \defword{symplectic group} is defined by
\[ \Sp_n(R) = \Set{X \in \GL_{2n}(R)}{\overline{X}^T J_n X = J_n} \subseteq \GL_{2n}(R) \subseteq \M_{2n}(R) \] %Poor
% wegen Adjunktion. siehe Martin 2013-01-14
% Tippfehler in Poor: Dort sind es n*n Matrixen, es müssen aber 2n*2n Matrixen sein.
where $J_n := \SmallMatrix{0}{-1_n}{1_n}{0} \in \SL_{2n}(R)$ (as in \cite{Dern01Herm}). (Note that some authors (e.g. \cite{PoorYuen07Comp}) define $J_n$ negatively.)
% In Dern ist U_2(\curlO) = \Sp_2(\curlO).
% Dern nennt das, und wir auch: Hermitian modular group (.. wo?)
% U steht für unitäre Gruppe.
%\[ U_n(R) = \Set{X \in \GL_{2n}(R)}{\overline{X}^T J_n X = J_n} \]
% Sp steht für symplectische Gruppe.
$\Sp_n(R)$ is also called the \defword{unitary group}. Note that \cite{Dern01Herm} uses $\operatorname{U}_n(R) = \Sp_n(R)$.
% Weil-Darstellung ist eine gewisse Darstellung von symplektischen Gruppen. Aber irrelevant für mich.
% In dem Zusammenhang waren auch vektorwertigen Modulformen, die auch nicht relevant für mich sind.
Also note that $M = \SmallMatrix{a}{b}{c}{d} \in \Sp_1(\Z)$ $\Leftrightarrow$ $ad - bc = 1$ $\Leftrightarrow$ $M \in \SL_2(\Z)$. Thus, $\Sp_1(\Z) = \SL_2(\Z)$. % Notiz 5.4.13

In addition, for a ring $R \subseteq \C$, define
\begin{align*}
\Rot(U) & = \SimpleMatrix{\overline{U}^T}{}{}{U^{-1}} \in \Sp_2(R), & U \in \GL_2(R) \\
\Trans(H) & = \SimpleMatrix{1_2}{H}{}{1_2} \in \Sp_2(R),  & H \in \Her_2(R)
\end{align*}
and note that we have $J_2 = \SmallMatrix{}{-1_2}{1_2}{} \in \Sp_2(R)$. Those tree types of matrices form a generator set for the group $\Sp_2(R)$.

For $Z \in \M_n(\C)$, we call
\[ \Re(Z) = \frac{1}{2} \left(Z + \overline{Z}^{T}\right) \in \M_n(\C) \]
the \defword{real} part and
\[ \Im(Z) = \frac{1}{2i} \left(Z - \overline{Z}^{T}\right)  \in \M_n(\C) \]
the \defword{imaginary} part of $Z$ and we have $Z = \Re(Z) + i \Im(Z)$.
Note that we usually have $\Re(Z),\Im(Z) \not\in \M_n(\R)$ but we have $\Re(Z),\Im(Z) \in \Her_n(\C)$.

We say that some function $f \colon \mathcal{A} \rightarrow \mathcal{B}$ with $\mathcal{A} \subseteq \M_n(R)$, $\mathcal{B} \subseteq R$ is \defword{$k$-invariant} under some $\mathcal{X} \subseteq \M_n(R)$ where $\mathcal{A}[\mathcal{X}] \subseteq \mathcal{A}$ if and only if $\det(U)^k f(T[U]) = f(T)$ for all $T \in \mathcal{A}$, $U \in \mathcal{X}$.

%Let $S$ be a set with $G$-action. Then the set of $G$-invariants $S^G$ is the set of all $s \in S$ satisfying $g s = s$ for all $G$.  We can equip the set of functions $\F \rightarrow \C$ with the action
%$( g f ) (T) = det(g)^k f( T[g] )$  % I'm not sure about the sign of $k$, even though this doesn't matter.
%and this lead to the definition that we need.


\subsection{Elliptic modular forms}
\defword{Elliptic modular forms} are holomorphic functions over the set
\[ \SiegelHalfPlane_1 := \Set{z \in \C}{\Im(z) > 0} \subseteq \C \]
which is called the \defword{Poincaré upper half plane}.

Let $f$ be a holomorphic function $\SiegelHalfPlane_1 \rightarrow \C$. Let $\Gamma$ be a subgroup of $\Sp_1(\Z) = \SL_2(\Z)$. \defword{Modular forms} are functions which are invariant with regard to a specific \defword{translation}. In this case, the translation is given by some $M \in \Gamma$ and a \defword{weight} $k \in \Z$. We also call $\Gamma$ the \defword{translation group}.

Let $M = \SmallMatrix{a}{b}{c}{d} \in \Gamma$ and $\tau \in \SiegelHalfPlane_1$. We write
\[ M \tau := \frac{a \tau + b}{c \tau + d} . \]
We define the \defword{translated function} $f | M \colon \SiegelHalfPlane_1 \rightarrow \C$ as
\[ (f | M) (\tau) :=  (c \tau + d)^{-k} \cdot f(M \tau) . \]

An \defword{Elliptic modular form} with weight $k \in \Z$ over $\Gamma$ is a holomorphic function
\[ f \colon \SiegelHalfPlane_1 \rightarrow \C \]
with
\begin{align*}
(1) \ \ & f | M  = f \quad \forall \ M \in \Gamma, \\
(2) \ \ & f(\tau) = O(1) \quad \text{for } \tau \rightarrow i \infty .
\end{align*}

Thus, (1) yields the equation
\[ f\left(\frac{a \tau + b}{b \tau + c}\right) = (c \tau + d)^k \cdot f(\tau) \quad \forall \ \SmallMatrix{a}{b}{c}{d} \in \Gamma, \tau \in \SiegelHalfPlane_1 . \]

$\ESpace{k}{\Gamma}$ denotes the vector space of such Elliptic modular forms.

In this work, we use a specific subgroup of $\Sp_1(\Z)$. We define
\[ \Gamma_0(l) := \Set{\SimpleMatrix{a}{b}{c}{d} \in \Sp_1(\Z)}{c \equiv 0 \pmod{l}} \subseteq \Sp_1(\Z) \subseteq \M_2(\Z) \]
as a subgroup of $\Sp_1(\Z)$.

An \defword{Elliptic modular cusp form} is an Elliptic modular form $f \colon \SiegelHalfPlane_1 \rightarrow \C$ with
\[ \lim_{t \rightarrow \infty} f(i t) = 0 . \]

We can represent the cusps with $\Gamma \backslash \Q$.

More general cusps: $\Gamma \backslash \SL_2(\Q) \div \Gamma_{\infty, \Q}$, where $\Gamma_{\infty, \Q}$ are the upper triangular matrices in $\GL_2(\Z)$.


\subsection{Siegel modular forms}

\defword{Siegel modular forms} are a generalization of Elliptic modular forms for higher dimensions.
Let
\[ \SiegelHalfPlane_n := \Set{Z \in \M_n^T(\C)}{\Im(Z) > 0} \]
be the \defword{Siegel upper half space}.
We call $\Sp_n(\Z)$ the \defword{Siegel modular group}.
Siegel modular forms are holomorphic functions $\SiegelHalfPlane_n \rightarrow \C$ for a given \defword{degree} $n \in \N$.

The \defword{translation group} $\Gamma$ is a subgroup of $\Sp_n(\Z)$. For $M = \SmallMatrix{A}{B}{C}{D} \in \Gamma$ and $Z \in \SiegelHalfPlane_n$, we write
%with $Z = S \tau$. https://mail.google.com/mail/u/0/#label/Diplomarbeit/13b471d95eb713e9
\[ M \cdot Z := (A Z + B) \cdot (C Z + D)^{-1} . \]
Generalizing the Elliptic translation, the Siegel \defword{translated function} $f | M \colon \SiegelHalfPlane_n \rightarrow \C$ is defined as   
\[ ( f | M ) (Z) :=
\det(CZ + D)^{-k} \cdot f(M \cdot Z) \]


% Groß/Kleinschreibung:
% Hermitian modular form
% Siegel modular cusp form
% In Überschriften fast alles (alle Nomen) kapitalisiert

% Skript Krieg, p.49. Aber die folgende Def ist woanders her, glaub ich...?
% cusp = Spitzenform
A \defword{Siegel modular form} of degree $n\in\N$ with weight $k \in \Z$ over $\Gamma$ is a holomorphic function
\[ f \colon \SiegelHalfPlane_n \rightarrow \C \]
with
\begin{align*}
(1) \ \ & f|M = f \quad \forall \ M \in \Gamma, \\
(2) \ \ & \text{for } n = 1 \colon f(Z) = O(1) \quad \text{for } Z \rightarrow i \infty
\end{align*}

$\SSpaceN{n}{k}{\Gamma}$ denotes the vector space of such Siegel modular forms.

Note that Elliptic modular forms are Siegel modular forms of degree $n=1$. Thus we have $\ESpace{k}{\Gamma} = \SSpaceN{1}{k}{\Gamma}$.

Siegel modular forms aren't directly used in this work. However, the idea of this work is inspired by \cite{PoorYuen07Comp} and they are using them.


\subsection{Hermitian modular forms}

Let
\[ \HalfPlane_n :=  \Set{ Z \in \M_n(\C) }{ \Im(Z) > 0} \]
be the \defword{Hermitian upper half space}. Note that these matrices are not symmetric as the Siegel upper half space $\SiegelHalfPlane_n$ but we have $\SiegelHalfPlane_n \subseteq \HalfPlane_n$ and $\SiegelHalfPlane_1 = \HalfPlane_1 \subseteq \C$.

\defword{Hermitian modular forms} are holomorphic functions $\HalfPlane_n \rightarrow \C$. They are a generalization of Siegel modular forms where the \defword{translation group} $\Gamma$ is not a subgroup of $\Sp_n(\Z)$ but a subgroup of $\Sp_n(\curlO)$ for some $\curlO \subseteq \C$.

% Def 1.8 bei Dern
% Bei Dern: j((A B C D),Z) = det(CZ + D)
% Multiplikatorsystem -> multiplier

More specificially,
let $\Delta \in \N$ so that we have the imaginary quadratic number field $\Q(\sqrt{-\Delta})$ where $-\Delta$ is the fundamental discriminant.
Then, let $\curlO \subseteq \Q(\sqrt{-\Delta})$ be the maximum order.
We call $\Sp_n(\curlO)$ the \defword{Hermitian modular group}.
Let $\Gamma$ be a subgroup of $\Sp_n(\curlO)$.

Again, with $M = \SmallMatrix{A}{B}{C}{D} \in \Gamma$, $Z \in \HalfPlane_n$, $M \cdot Z := (A Z + B) \cdot (C Z + D)^{-1}$ as for Siegel modular forms and the \defword{weight} $k \in \Z$, we define the \defword{translated function} $f | M \colon \HalfPlane_n \rightarrow \C$ as
\[ (f|M) (Z) := \det(CZ + D)^{-k} \cdot f(M \cdot Z) .\]

A \defword{Hermitian modular form}
of \defword{degree} $n\in\N$
with \defword{weight} $k\in \Z$
over $\Gamma$
is a holomorphic function
\[ f \colon \HalfPlane_n \rightarrow \C \]
with
\begin{align*}
(1) \ \ & f | M = f \quad\forall\ M \in \Gamma, Z \in \HalfPlane_n , \\
(2) \ \ & \text{for } n = 1 \colon \ \text{$f$ is holomorphic in all cusps} .
\end{align*}

$\HSpaceN{n}{k}{\Gamma}$
denotes the vector space of such Hermitian modular forms.

As it can be done for Siegel modular forms, we generalize this further by introducing a \defword{Multiplicative character} $\nu \colon \Gamma \rightarrow \C^\times$. Thus, for $M_1, M_2 \in \Gamma$, we have $\nu(M_1) \cdot \nu(M_2) = \nu(M_1 \cdot M_2)$.

A \defword{Hermitian modular form} over $\Gamma$ and $\nu$
is a holomorphic function
\[ f \colon \HalfPlane_n \rightarrow \C \]
with
\begin{align*}
(1) \ \ & f | M = \nu(M) \cdot f \quad\forall\ M \in \Gamma, Z \in \HalfPlane_n , \\
(2) \ \ & \text{for } n = 1 \colon \ \text{$f$ is holomorphic in all cusps} .
\end{align*}

%$[\Gamma, k, \nu]$ <- früher und Dern, aber jetzt:
$\HSpaceN{n}{k}{\Gamma,\nu}$
denotes the vector space of such Hermitian modular forms.

In this work, we will always use Hermitian modular forms of degree $n=2$.
%We will start with $\Delta \in \Set{3,4,8}$.
% weight k is fixed in the algorithm, but the algorithm allows any k.

\subsubsection{Properties}
% Wikipedia: http://en.wikipedia.org/wiki/Fundamental_discriminant
% Bei WP ist D = -\Delta, deshalb sind die Werte anders!
% Wir haben es wie bei Dern.
Because $-\Delta$ is fundamental, we have two possible cases:
\begin{enumerate}
\item $\Delta \equiv 3 \pmod{4}$ and $\Delta$ is square-free, or
\item $\Delta \equiv 0 \pmod{4}$, $\Delta/4 \equiv 1,2 \pmod{4}$ and $\Delta/4$ is square-free.
\end{enumerate}
\label{maxorder}
And for the \defword{maximum order} $\curlO$, we have (compare \cite{Dern01Herm})
\begin{align*}
\curlO = &\ \Z +  \Z \frac{-\Delta+i\sqrt{\Delta}}{2} , \\
\curlO^\# = & \ \Z \frac{i}{\sqrt{\Delta}} + \Z \frac{1 + i\sqrt{\Delta}}{2} .
\end{align*}

From now on, we will always work with Hermitian modular forms of degree $n=2$. We also use $\Gamma = \Sp_2(\curlO)$ for simplicity.


%!TEX root =  index.tex

\section{Theory}


% TODO: wofür brauchen wir das?
\begin{lemma}
Let $f \colon \M_2(\C) \rightarrow \C$ be a Hermitian modular form of weight $k$. Let $S \in \PM_2(\C)$.
Then, $f(S \tau) \colon \HalfPlane_1 \subseteq \C \rightarrow \C$ is an elliptic modular form of weight $2k$ to $\Gamma_0(l)$, where $l$ is the denominator of $S^{-1}$.  %for some matrix $S \in \M_2(\Z)$ with $\Gamma(S) \subseteq \SL_2(\Z)$.
\end{lemma}


% TODO: wofür brauchen wir das?
% von Notizen im Block
% Prop 7.3 von Poor für herm Modulformen
\begin{lemma}
Prop 7.3. von Poor für herm Modulformen.
$\Gamma(\mathcal{L}) \supseteq \Gamma_0(l)$ for $l \in \Z^+, ls^{-1} \in \mathcal{P}_n(\curlO)$.
% \Gamma ist kein Gitter, sondern eine diskrete Untergruppe
% Beweisskizze in Unterlagen
% L symmetrische, ganze Matrix => \Gamma(L) \subseteq \SL{2}(\ZZ)
% \curlL ist ein polarisiertes Gitter, welches wir zur Vereinfachung als gewöhliche symmetrische, ganz Matrix ansehen
\end{lemma}


We want to calculate a generating set for the Fourier expansions of Hermitian modular forms. Now we will formulate the main algorithm of our work.

\begin{algo}
We have the Hermitian modular form degree $n = 2$ fixed, as well as some $\Delta$ (for now, $\Delta \in \Set{3,4,8}$). Then we select some form weight $k \in \Z$ ($k \in \Set{1,\dots,20}$ or so), some $\curlO \subseteq \Q(\sqrt{-\Delta})$ and some subgroup $\Gamma$ of $\Sp_2(\curlO)$. Then we select an abel character $\nu \colon \Gamma \rightarrow \C^\times$ of $\Sp_2(\curlO)$.

We define the index set
\[ \Lambda := \Set{0 \le \SimpleMatrix{a}{b}{\overline b}{c} \in \M_2(\curlO^\#)}{a,c \in \Z } . \]

We start with $l = 1$ and increase it but only use the square-free numbers.

Fix $B \in \N$ as a limit. Select a precision
\[ \F := \Set{\SimpleMatrix{a}{b}{\overline b}{c}}{0 \le a , c < B, b \in \curlO^\#} \subseteq \Lambda . \]
% \Lambda sind die die Indizes der Fourier-Entwicklung einer Hermitischen Modulform.

\begin{enumerate}
\item Set $\mathcal{S} = \{\}$,
\item Enumerate matrices $S \in \M_2^T(\Z)$, and set $\mathcal{S} \leftarrow \mathcal{S} \cup \{ S \}$ and for each time you add a new matrix perform the following steps.

%Select a set of matrices $\mathcal{S} \subseteq \M_2^T(\Z)$ with $0 < S \in \mathcal{S}$.
%Make $\mathcal{S}$ big enough.
%Now, for some $S \in \mathcal{S}$:

\item
\[ \mathcal{M}_{k,\mathcal{S},\F}^H = \Set{ (f [S])_{S\in\mathcal{S}} }{\text{$f \in \Q^\F$ is $\GL_2(\curlO)$ invariant}} \subseteq \bigoplus_S \Q^{\F(S)} , \]
% unter GL_2(\curlO) invariant:
% also Fourier-Entwicklung unter der Operation von \GL{2}(\cO) invariant:
% also wenn a(T), T \in \Lambda die Fourier-Koeffizienten bezeichnen, dann gilt \det(U)^k a(T[U]) = a(T) für alle U \in \GL{2}(\cO)
\[ \mathcal{M}_{k, \mathcal{S}} = \bigoplus_S \FE_{\F(S)}(\M_k(\Gamma(l_S))) \]

\item
If
\[ \dim \mathcal{M}_{k,\mathcal{S},\F}^H \cap \mathcal{M}_{k, \mathcal{S}}
= \dim M^H_k , \]
then we are ready and we can reconstruct the Fourier expansion in the following way: ...

If not, then return to Step 2, and enlarge $\mathcal{S}$.
\end{enumerate}
\end{algo}

%Matrix-repr of f \in (\Q^{\CurlfF[G]})^G_Chi \mapsto (\sum_{tr(ST)=n} a(T))_{n,S}
%( )^{G, chi} meine ich die Menge der Elemente von Q^\cF[G], die a(T[g]) = chi(g) a(T) erfüllen
% Eine Basis davon kannst du dir mittels der Reduktionen von T überlegen

%!TEX root =  index.tex

\section{Implementation}
\label{chapter:impl}

In this chapter, we are describing the implementation. All of the code can be found at \cite{Zeyer13Github}.

The code consists of several parts.
All of it was implemented around the Sage (\cite{sage}) framework, thus the main language is Python (\cite{python}).
For performance reasons, some very heavy calculations have been implemented in C++ (\cite{cpp}) and some Cython (\cite{cython}) code is the interface between both parts.

The implementation is complete as far as what we have developed in this thesis. Unfortunately, at the time of writing, we haven't gotten any results yet. Many parts of the code have been tested in various way but it seems likely that there are still bugs. More details can be seen in the specific sections and in the code.

\subsection{Code structure}

We introduce the most important files and details. Other files might be mentioned elsewhere.

\subsubsection{Main function \ifuncname{herm_modform_space}}
The main entry point is in the file \ifilename{algo.py}.
The function \ifuncname{herm_modform_space} calculates the Hermitian modular forms space.
The function gets the fundamental discriminant \isymbname{D} $= - \Delta$, the Hermitian modular forms weight $k =$ \isymbname{HermWeight} and the precision limit $B_\F =$ \isymbname{B_cF} as its input and returns the vector space of Fourier expansions of Hermitian modular forms to the precision $B_\F$. The Fourier expansions are indexed by the reduced matrices of $\F$ (see \cref{remark:reducedCurlF} for details). This index list can also be returned by \ifuncname{herm_modform_indexset}.

The function can also do its calculation in parallel via multiple processes. As a convenience, to easily start the calculation with $N$ processes in parallel, there is the function \ifuncname{herm_modform_space__parallel} with the additional parameter \isymbname{task_limit}, where you just set \isymbname{task_limit} $=N$. For details about the parallelization, see \cref{impl:parallelization}.

Thus, to calculate the Hermitian modular forms with $D=-3$, weight $6$ and $B_\F = 7$, you can do:
\begin{lstlisting}
# run sage in the `src` directory of this work
import algo
algo.herm_modform_space(D=-3, HermWeight=6, B_cF=7)
\end{lstlisting}
Or, if you want to use 4 processes in parallel:
\begin{lstlisting}
algo.herm_modform_space__parallel(
    D=-3, HermWeight=6, B_cF=7, task_limit=4)
\end{lstlisting}

\subsubsection{\ifilename{algo.py}}
The function \ifuncname{herm_modform_space} uses \ifuncname{modform_restriction_info} and \ifuncname{modform_cusp_info} which are also defined in the same file.
The theory behind these functions is described in \cref{ellipticReduction} and \cref{cuspInfo}, accordingly. Details about the implementation are also described in \cref{calcMatrix} and \cref{calcMatrixTrans}, accordingly.

Both return a vector space which is a superspace of the Hermitian modular form Fourier expansions and \ifuncname{herm_modform_space} intersects them until the final dimension is reached.

\subsubsection{\ifilename{helpers.py}}
This file contains many other mathematical calculations needed by \ifilename{algo.py}. These are, among others:
\begin{itemize}
\item Calculations in $\curlO$ and $\curlO^\#$ (as described in \cref{curlOcalcs}),
\item \ifuncname{solveR} (as described in \cref{solveR}),
\item \ifuncname{xgcd} and \ifuncname{divmod} (as described in \cref{xgcd}),
\item some wrappers around \ifuncname{calcMatrix} from \ifilename{algo_cpp.cpp} and others,
\item some reimplementations of the C++ code for demonstration and testing.
\end{itemize}

\subsubsection{\ifilename{algo_cpp.cpp}, \ifilename{structs.hpp} and other C++/Cython code}
These files contains all the heavy calculation code. For example, these are
\begin{itemize}
\item Again, calculations in $\curlO$ and $\curlO^\#$ (as described in \cref{curlOcalcs}),
\item the iteration of $\F$ (as described in \cref{curlFiteration}),
\item the iteration of $S \in \PM_2(\curlO)$ (as described in \cref{Siter}),
\item \ifuncname{reduceGL} (as described in \cref{reduceGL}),
\item \ifuncname{calcMatrix} (as described in \cref{calcMatrix}),
\item \ifuncname{calcMatrixTrans} (as described in \cref{calcMatrixTrans}).
\end{itemize}

\subsubsection{\ifilename{checks.py}}
In some calculations, such as \ifuncname{modform_restriction_info} in \ifilename{algo.py}, it is possible to do some checks whether the intermediate calculations are sane. For example, in some cases, we can check some properties which must hold for all superspaces of $\FE(\HSpace{k}{\Gamma})$.

\subsubsection{\ifilename{utils.py}}
This file contains mostly non-mathematical related utilities.
\begin{itemize}
\item It contains an extended \isymbname{Pickler} which overcomes some problems with the default \isymbname{Pickler}. Otherwise, some Sage objects would not be serializable. Also, serialization becomes deterministic.
\item There are several functions for persistent on-disk caching via serialization, such as \isymbname{PersistentCache} which is a dictionary which saves each entry in a separate file which makes Git-merging easier.
\item It also contains all the utilities for the parallelization. Details are described in \cref{impl:parallelization}.
\end{itemize}

\subsubsection{\ifilename{tests.py}}
This file contains some tests for some of the functions in \ifilename{algo.py}, \ifilename{helpers.py} and \ifilename{utils.py}. It is not needed otherwise.

%---
\

In the rest of this chapter, we will demonstrate the details of the calculations and representations.


\subsection{$\curlO$ and $\curlO^\#$ representation and calculations}
\label{curlOcalcs}

To represent $\curlO$ and $\curlO^\#$ in code, mostly in the low level C++ code (files \ifilename{algo_cpp.cpp}, \ifilename{structs.hpp}, \ifilename{reduceGL.hpp}), we can use two integers in both cases as the coefficients of some basis.

Most of the calculations presented in this section are implemented in \ifilename{structs.hpp}.

\subsubsection{Representations}

%\paragraph{$\curlO$.}
\label{impl:repr:curlO}

For $a \in \curlO$, we use
\[ a = a_1 + a_2 \frac{D + \sqrt{D}} {2} \]
with $a_1,a_2 \in \Z$.
It holds
\begin{align*}
\Re(a) = &\; a_1 + a_2 \frac{D}{2} , \\
\Re(a)^2 =&\; a_1^2 + D a_1 a_2 + \frac{D^2}{4} a_2^2 , \\
\Im(a) =&\; a_2 \frac{\sqrt{-D}}{2} , \\
\Im(a)^2 =&\; a_2^2 \frac{-D}{4} , \\
|a|^2 =&\; \Re(a)^2 + \Im(a)^2 = a_1^2 - (-D) a_1 a_2 + \frac{D^2-D}{4} a_2^2 .
\end{align*}
Note that $4$ divides $D^2 - D$. Thus, $|a|^2 \in \Z$.

%4.5.13. b\in\curlO
Sometimes we have given $a \in \K$ where we easily have $\Re(a)$ and $\Im(a)$ available and we want to calculate $a_1, a_2 \in \Q$ in the above representation. We get
\begin{align*}
a_2 = &\; \Im(a) \frac{2}{\sqrt{-D}}, \\
a_1 = &\; \Re(a) - a_2 \frac{D}{2} = \Re(a) + \Im(a) \sqrt{-D} .
\end{align*}

%\paragraph{$\curlO^\#$.}
\label{impl:repr:curlOdual}

For $b \in \curlO^\#$, we use
\[ b = b_1 \frac{1}{\sqrt{D}} + b_2 \frac{1 + \sqrt{D}} {2} \]
with $b_1,b_2 \in \Z$.
% 6.6. alt det \cO^#
It holds
\begin{align*}
\Re(b) = &\; \frac{1}{2} b_2, \\
\Re(b)^2 = &\; \frac{1}{4} b_2^2, \\
\Im(b) = &\; -\frac{b_1}{\sqrt{-D}} + \frac{1}{2} \sqrt{-D} b_2, \\
\Im(b)^2 = &\; \frac{b_1^2}{-D} - b_1 b_2 + \frac{1}{4} (-D) b_2^2, \\
|b|^2 = &\; \Re(b)^2 + \Im(b)^2 = \frac{b_1^2}{-D} - b_1 b_2 + \frac{1}{4} (1-D) b_2^2 .
\end{align*}
When we need $|b|^2$ in an implementation, we can multiply it with $-D$ to get an integer:
\[ (-D) |b|^2 = b_1^2 - (-D) b_1 b_2 + \frac{D^2-D}{4} b_2^2 . \]
%
%4.5.13 b \in \curlO^#
%5.6.13 a \in \curlO^#
When we have $b \in \K$ where $\Re(b)$ and $\Im(b)$ are easily available and when we want to calculate $b_1,b_2 \in \Q$ in the above representation, we get
\begin{align*}
b_2 =&\; 2 \Re(b) , \\
b_1 =&\; b_2 \frac{-D}{2} - \Im(b) \sqrt{-D} = \Re(b) (-D) - \Im(b) \sqrt{-D} .
\end{align*}

%3.5.13 \curlO^# conjugate
Let us calculate the complex conjugate $\overline{b}$ of $b \in \curlO^\#$:
\begin{align*}
\overline{b} &= \frac{-b_1}{\sqrt{D}} + \frac{b_2}{2} - b_2 \frac{\sqrt{D}}{2} \\
&\overset{!}{=} \hat{b}_1 \frac{1}{\sqrt{D}} + \hat{b}_2 \frac{1 + \sqrt{D}} {2} \\
\Rightarrow \quad \hat{b}_2 &= b_2 , \\
\hat{b}_1 &= \overline{b} \sqrt{D} - \hat{b}_2 (\sqrt{D}+D) \tfrac{1}{2} \\
&= b_2 \frac{\sqrt{D}}{2} - b_2 \frac{\sqrt{D}}{2} - b_2 \frac{D}{2} - b_2 \frac{D}{2} - b_1 \\
&= -b_2 D - b_1 .
\end{align*}

%3.5.13 \curlO^# conjugate
Note that $b \in \R$ if and only if $b_1 \frac{1}{\sqrt{D}} = - b_2 \frac{\sqrt{D}}{2}$, i.e.
\[ 2 b_1 = - b_2 D . \]

\subsubsection{Multiplications}
\label{curlOmultiplications}
%3.5. ElemOfCurlO.mul()
Let $a,b \in \curlO$ with $a = a_1 + a_2 \frac{D + \sqrt{D}} {2}$, $b = b_1 + b_2 \frac{D + \sqrt{D}} {2}$. Then we have
\begin{align*}
a \cdot b &= a_1 b_1 + a_1 b_2 (D + \sqrt{D}) \tfrac{1}{2} + b_1 a_2 (D + \sqrt{D}) \tfrac{1}{2}
+ a_2 b_2 \tfrac{1}{4} \underbrace{(D^2 + 2 D \sqrt{D} + D)}_{= 2D (D + \sqrt{D}) - D^2 + D} \\
&= \frac{\sqrt{D} + D}{2} (a_1 b_2 + b_1 a_2 + D a_2 b_2)
+ a_1 b_1 - a_2 b_2 \frac{D^2 - D}{4} .
\end{align*}

%3.5.13 mult \curlO^# und \curlO
Now, let $a \in \curlO^\#$ and $b \in \curlO$ with
\begin{align*}
a &= a_1 \frac{1}{\sqrt{D}} + a_2 \frac{1 + \sqrt{D}} {2} , \\
b &= b_1 + b_2 \frac{D + \sqrt{D}} {2} .
\end{align*}
Then we have
\begin{align*}
a \cdot b &= a_1 b_1 \tfrac{1}{\sqrt{D}} + a_1 b_2 (\sqrt{D} + 1) \tfrac{1}{2}
+ a_2 b_1 (1 + \sqrt{D}) \tfrac{1}{2} + a_2 b_2
\underbrace{(D + \sqrt{D} + D \sqrt{D} + D)}_{
\begin{aligned}
= 2D + \sqrt{D} + D \sqrt{D} \\
= 2D + \sqrt{D} (1 + D)
\end{aligned}
}
\tfrac{1}{4} \\
&= a_1 b_1 \tfrac{1}{\sqrt{D}} + (a_1 b_2 + a_2 b_1) (1 + \sqrt{D}) \tfrac{1}{2}
+ a_2 b_2 (2D + \sqrt{D}(1 + D)) \tfrac{1}{4} .
\end{align*}
Thus, when representing $a \cdot b \in \curlO^\#$ as
\[ a \cdot b = (ab)_1 \frac{1}{\sqrt{D}} + (ab)_2 \frac{1 + \sqrt{D}} {2} , \]
we get
\[ (ab)_2 = a_1 b_2 + a_2 b_1 + a_2 b_2 D \]
and
\begin{align*}
(ab)_1 &= \sqrt{D} ab - (ab)_2 (\sqrt{D} + D) \tfrac{1}{2} \\
&= a_1 b_1 + (a_1 b_2 + b_1 a_2) (\sqrt{D} + D) \tfrac{1}{2} + a_2 b_2 (D + \sqrt{D})^2 \tfrac{1}{4} \\
&\quad - (a_1 b_2 + a_2 b_1 + a_2 b_2 D) (\sqrt{D} + D) \tfrac{1}{2} \\
&= a_1 b_1 + a_2 b_2 \underbrace{( (D+\sqrt{D})^2 \tfrac{1}{4} - D(\sqrt{D}+D)\tfrac{1}{2} )}_{
\begin{aligned}[l]
=& \tfrac{D^2}{4} + \tfrac{D\sqrt{D}}{2} + \tfrac{D}{4} - \tfrac{D\sqrt{D}}{2} - \tfrac{D^2}{2} \\
=& \tfrac{D^2 - D}{4}
\end{aligned}
} \\
&= a_1 b_1 + a_2 b_2 \frac{D^2 - D}{4} .
\end{align*}

\subsubsection{Determinant of 2-by-2 matrices}
\label{detCurlO}
%16.4.13 det(S)
For $[a,b,c] \in \Her_2(\C)$, we have
\[ \det([a,b,c]) = ac - b \overline{b} = ac - |b|^2 . \]
%
When we have $b \in \curlO$ or $b \in \curlO^\#$, we have given a formula for $|b|^2$ in \cref{impl:repr:curlO}. With those representations and $a,c \in \Z$, for $b \in \curlO$, we have
\[ \det([a,b,c]) = a c -  b_1^2 + (-D) b_1 b_2 - \tfrac{D^2-D}{4} b_2^2 \in \Z \]
and for $b \in \curlO^\#$, we have
\[ \det([a,b,c]) = a c - b_1^2 \tfrac{1}{-D} + b_1 b_2 - \tfrac{1}{4} (1-D) b_2^2 \in \tfrac{1}{-D} \Z . \]

In the code, we represent both matrices $\Her_2(\curlO)$ and $\Her_2(\curlO^\#)$ by 4-tuples $(a,b_1,b_2,c) \in \Z^4$.

\subsubsection{Trace of $TS$}
\label{traceST}
%16.4.13 tr(ST)
We want to calculate $\tr(TS)$ for $T \in \Her_2(\curlO^\#)$, $S \in \Her_2(\curlO)$.
Let $T = [T_a, T_b, T_c]$ and $S = [S_a, S_b, S_c]$ with
\begin{align*}
T_b &= T_{b1} \frac{1}{\sqrt{D}} + T_{b2} \frac{1 + \sqrt{D}} {2} , \\
S_b &= S_{b1} + S_{b2} \frac{D + \sqrt{D}} {2}
\end{align*}
and we have
\[ \overline{S_b} = S_{b1} + S_{b2} \frac{D - \sqrt{D}} {2} . \]
Then,
\[
\tr(T S) = T_a S_a
+ \underbrace{T_b \overline{S_b} + \overline{T_b} S_b}_{= 2 \Re(T_b \overline{S_b})}
+ T_c S_c
\]
and
\begin{align*}
\overline{S_b} T_b &= S_{b1} T_{b1} \tfrac{1}{\sqrt{D}} + S_{b1} T_{b2} (1 + \sqrt{D}) \tfrac{1}{2}
+ S_{b2} D \tfrac{1}{2} T_{b1} \tfrac{1}{\sqrt{D}}
- S_{b2} \tfrac{1}{2} T_{b1} \\
&\quad + T_{b2} S_{b2} \tfrac{1}{4} \underbrace{(D - \sqrt{D} + D \sqrt{D} - D)}_{=\sqrt{D}(D-1)} \\
\Rightarrow \Re(\overline{S_b} T_b) &= S_{b1} T_{b2} \tfrac{1}{2} - S_{b2} T_{b1} \tfrac{1}{2} .
\end{align*}
Thus, in our computer implementation, we can just use
\[ \tr(T S) = T_a S_a + T_c S_c + S_{b1} T_{b2} - S_{b2} T_{b1} . \]
And if we have $T_a, T_{b1}, T_{b2}, T_{c}, S_a, S_{b1}, S_{b2}, S_c \in \Z$, we also have $\tr(TS) \in \Z$.


\subsection{Iteration of the precision Fourier indice $\F$}
\label{curlFiteration}

The set $\F$ depends on a limit $B_\F \in \N$:
\[ \F = \F_B = \Set{\SimpleMatrix{a}{b}{\overline b}{c} \in \Lambda}{0 \le a , c < B_{\F}} \subseteq \Lambda . \]
In \cref{remark:reducedCurlF}, we see that $\F$ is finite.

We have implemented an iteration of $\F$ in a way that the list of $\F_{B_2}$ always starts with $\F_{B_1}$ if $B_1 \le B_2$. That is \ifuncname{PrecisionF} in \ifilename{algo_cpp.cpp}. For testing and demonstration purpose, there is also a pure Python implementation \ifuncname{curlF_iter_py} in \ifilename{helpers.py}. I.e., in Python, for some $D$ and $B1 \le B2$, it yields:
\begin{lstlisting}
curlF1 = list(curlF_iter_py(D=D, B_cF=B1))
curlF2 = list(curlF_iter_py(D=D, B_cF=B2))
assert curlF1 == curlF2[:len(curlF1)]
\end{lstlisting}

The algorithm of the iteration of $T \in \F$ works in the following way: We have the current matrix represented as integers $a,b_1,b_2,c \in \Z$ and we start with each of them set to $0$. Then, $b = b_1 \frac{1}{\sqrt{D}} + b_2 \frac{1 + \sqrt{D}} {2}$ and $T = [a,b,c]$. We have the limit $B_\F \in \N_0$ and iterate an internal limit $\tilde{B} \in \Set{0,1,\dots,B_\F-1}$.
\begin{enumerate}
\item If the current saved matrix is a valid one, i.e. its determinant is not negative and $0 \le a,c \le B_\F$, we return it.
\item We iterate $b_2$ through $\Set{0,1,-1,2,-2,\dots}$.
\item
%23.5.13 \F iteration fix (\curlO^#). b_2
The absolut limit for $b_2$ is given by
\[ 4 a c \ge b_2^2 . \]
\begin{proof}
With $\det(T) \ge 0$, we have
\[ (-D) a c \ge b_1^2 - (-D) b_1 b_2 + \frac{(-D)(1-D)}{4} b_2^2 \]
(see \cref{detCurlO}).

And it yields
\[ b_1^2 - (-D) b_1 b_2 = (b_1 - \tfrac{-D}{2} b_2)^2 - \tfrac{D^2}{4} b_2^2 \ge - \tfrac{D^2}{4} b_2^2 , \]
thus
\[ (-D) a c \ge - \tfrac{D^2}{4} b_2^2 + \frac{(-D)(1-D)}{4} b_2^2 = \tfrac{-D}{4} b_2^2 . \]
This is equivalent with the inequality to-be-proved.
\end{proof}
\item Once we hit that limit, we reset $b_2 := 0$ and we do one iteration step for $b_1$ through the set $\Set{0,1,-1,2,-2,\dots}$.
\item
%23.5.13 \F iteration fix (\curlO^#). nr2. b_1
The absolut limit for $b_1$ is given by
\[ a c (D^2 - D) \ge b_1^2 . \]
\begin{proof}
We have
\[ \tfrac{D^2-D}{4} b_2^2 - (-D) b_1 b_2
= \left( \sqrt{\tfrac{D^2-D}{4}} b_2 - \frac{-D}{2 \sqrt{\tfrac{D^2-D}{4}}} b_1 \right)^2
- \tfrac{(-D)^2}{D^2 - D} b_1^2
\ge - \tfrac{D^2}{D^2 - D} b_1^2 .
\]
Then, again with $\det(T) \ge 0$ like in the limit for $b_2$, we have
\[ (-D) a c \ge b_1^2 - \frac{D^2}{D^2 - D} b_1^2 = b_1^2 \frac{-D}{D^2-D} . \]
This is equivalent with the inequality to-be-proved.
\end{proof}
\item Once we hit that limit, we reset $b_1 := b_2 := 0$ and we increase $c$ by one.
\item Once we hit $c > \tilde{B}$, we reset $b_1 := b_2 := 0$ and we increase $a$ by one, if $a < \tilde{B}$. For all cases where $a < \tilde{B}$, we set $c := \tilde{B}$, otherwise $c := 0$.
\item Once we hit $a = \tilde{B}$, we increase $\tilde{B}$ by one and reset $a := 0$ and $c := \tilde{B}$.
\item Once we hit $\tilde{B} \ge B_\F$, we are finished.
\end{enumerate}

We have seen in \cref{reducedCurlF} that it is sufficient to use $\invarF{\F}{\GL_2(\curlO)}$ as the index set. In our implementation, we iterate through $\F$ and save the first occurrence of a new reduced matrix in a list. That list is returned by the function \ifuncname{herm_modform_indexset} which is declared in \ifilename{helpers.py}. It uses the C++ implementation in \ifilename{algo_cpp.cpp} as its backend. For testing and demonstration purpose, there is also a pure Python implementation \ifuncname{herm_modform_indexset_py} in \ifilename{helpers.py}.


\subsection{Iteration of $S \in \PM_2(\curlO)$}
\label{Siter}

The matrices $S \in \PM_2(\curlO)$ are used for the restriction in $f[S]$ for an Hermitian modula form $f$ as described in \cref{ellipticReduction}.

There are multiple implementations of this infinite iteration. Our first version only iterated through reduced matrices $\PM_2(\Z)$ with increasing determinator. We want the increasing determinant because we want to exhaust all possible matrices with low determinants because they are easier for the rest of the calculations. Later, it turned out that matrices only over $\Z$ don't yield enough information and we need matrices with imaginary components. Once you add the imaginary component, it is not possible anymore to iterate through all of $\PM_2(\Z)$ with increasing determinant because there can be infinity many matrices for a given determinant (or it is not trivial to see if there are not and how to set the limits in an implementation). Thus, the second implementation for matrices over $\PM_2(\curlO)$ does not keep the determinant fixed and rather works very similar to the iteration through $\F$, as described in \cref{curlFiteration}.

The implementation is in C++ in the file \ifilename{algo_cpp.cpp}. The class \isymbname{CurlS_Generator} owns and manages the iterator and can store several matrices at once because the main matrix calculation implementations (\cref{calcMatrix} and \cref{calcMatrixTrans}) can be done for several matrices at once. The class \isymbname{M2T_O_PosDefSortedZZ_Iterator} implements the iteration through $\PM_2(\Z)$ with increasing denominator. The class \isymbname{M2T_O_PosDefSortedGeneric_Iterator} implements the generic iteration through $\PM_2(\curlO)$.

The infinite iteration through $S \in \PM_2(\Z)$ in \isymbname{M2T_O_PosDefSortedZZ_Iterator} works as follows: We represent $S$ as $a,b,c \in \Z$ with $[a,b,c] = S$. We start with each of them set to zero. Also, we internally save the current determinant $\delta$ and start with $\delta := 0$.
\begin{enumerate}
\item We return a matrix if it is valid and reduced. That means that we only return if $a \le c$, $\det([a,b,c]) = ac - b^2 = \delta$ and if there is no common divisor of $a,b,c$ except $1$.
\item We increase $c$ by one. We set $a := \lfloor \frac{\delta + b^2}{c} \rfloor$.
\item Once we hit $c > \delta + b^2$, we reset $c := 0$ and make one iteration step for
$b \in \Set{0,1,-1,2,-2,\dots}$.
\item
Once we hit $3 b^2 > \delta$, we know that there aren't any further matrices with this determinant $\delta$. Thus we reset $a := b := c := 0$ and increase $\delta$ by one.
% 8.4.13 \S iteration
\begin{proof}
For a reduced matrix $[a,b,c]$, we have
\[ 0 \le 2 |b| \le a \le c . \]
Thus,
\[ \delta = ac - b^2 \ge (2 |b|) (2 |b|) - b^2 = 3 b^2 . \qedhere \]
\end{proof}
\end{enumerate}

The infinite iteration through $S \in \PM_2(\curlO)$ in \isymbname{M2T_O_PosDefSortedGeneric_Iterator} is mostly the same as the iteration of $\F$ as described in \cref{curlFiteration}. The difference is that $\F$ is over $\curlO^\#$ and $S$ is over $\curlO$. This yields to other limits for $b_1$ and $b_2$. Also, we don't have a limit like $\tilde{B}$.
\begin{enumerate}
\item We iterate $b_2$ through $\Set{0,1,-1,2,-2,\dots}$.
\item Once we hit the absolut limit of $b_2$, we reset $b_2 := 0$ and make one iteration step for $b_1 \in \Set{0,1,-1,2,-2,\dots}$.
\item Once we hit the absolut limit of $b_1$, we reset $b_1 := b_2 := 0$ and increase $c$ by one.
\item Once we hit $c > a$, we reset $c := b_1 := b_2 := 0$ and increase $a$ by one.
\end{enumerate}
%
%23.5.13 \S iteration fix (\curlO)
The absolut limit of $b_2$ is given by
\[ 4 ac \ge (-D) b_2^2 \]
and the absolut limit of $b_1$ is given by
\[ ac (1 - D) \ge b_1^2 . \]
\begin{proof}
We have $\det(S) \ge 0$ and thus
(see \cref{detCurlO})
\[ a c \ge b_1^2 - (-D) b_1 b_2 + \frac{D^2 - D}{4} b_2^2 . \]
Note that this is mostly like the inequality in the case over $\curlO^\#$, except that we have $ac$ on the left side instead of $(-D) ac$. Thus, we can mostly reuse the $b_1, b_2$ limit calculations from \cref{curlFiteration}. For $b_1$, we have
\[ a c \ge b_1^2 \frac{-D}{D^2-D} . \]
This is equivalent to the inequality to-be-proved.
And for $b_2$, we have
\[ a c \ge \tfrac{-D}{4} b_2^2 . \qedhere \]
\end{proof}


\subsection{\ifuncname{reduceGL}}
\label{impl:reduceGL}
\label{reduceGL}
In \cref{remark:reducedCurlF}, we have described that it is sufficient to use reduced matrices $\hat{T} \in \F$. Thus, in our implementation, for a given matrix $T \in \F$, we need a way to calculate the reduced matrix $\hat{T} \in \F$ such that
\[ \hat{T}[U_T] = T \]
for some $U_T \in \GL_2(\curlO)$. In the code, we don't need $U_T$ directly but rather the determinant of $U_T$.

Dominic Gehre and Martin Raum have developed a Cython implementation \cite{Raum09reduceGL} of "Functions for reduction of fourier indice of Hermitian modular forms". This function \ifuncname{reduceGL} gets a matrix $T \in \Her_2(\curlO^\#)$ and returns the Minkowski-reduced matrix $\hat{T} \in \Her_2(\curlO^\#)$ and some character evaluation of $U_T$ which also declares the determinant of $U_T$.

In this work, this function \ifuncname{reduceGL} has been reimplemented in C++ (\ifilename{reduceGL.hpp}) and in Python (\ifilename{reduceGL.py}).


\subsection{\ifuncname{divmod} and \ifuncname{xgcd}}
\label{xgcd}

We have given numbers $a,b \in \curlO$ and we search for $d,p,q \in \curlO$ such that $d = pa + qb$ and $d$ divides $a$ and $b$. Then, $d$ is also the greatest common divisor (\ifuncname{gcd}). This is also equivalent to
\[ 1 = p \frac{a}{d} + q \frac{b}{d} . \]
%
For example, we need that in \ifuncname{solveR} (\cref{solveR}).

The extended Euclidean algorithm (\ifuncname{xgcd}) is the standard algorithm to calculate these numbers. It works over all Euclidean domains. In our case, it works for $\Delta \in \Set{1, 2, 3, 7, 11}$.

Sage has \ifuncname{xgcd} which works only for integers. It doesn't directly offer functions to calculate the \ifuncname{xgcd} over quadratic imaginary number fields.

Thus, in this work, we have reimplemented a simple canonical version of \ifuncname{xgcd} for $\curlO$ with a few fast paths, e.g. in the case of integers. This implementation can be found in the class \isymbname{CurlO} in \ifilename{helpers.py}.

The main work is done in the \ifuncname{divmod} function. \ifuncname{divmod} gets two numbers $a, b \in \curlO$ and returns $q, r \in \curlO$ such that $q b + r = a$. This is the division with remainder. It holds that $f(r) < f(b)$ for the Euclidean Norm $f \colon \K \rightarrow \R_{\ge 0}$. In our case, we have $f(x) = |x|$. The current implementation of \ifuncname{divmod} is very naive and should be improved. It can be found as well in the class \isymbname{CurlO} in \ifilename{helpers.py}.

Let $a,b \in \curlO$ with $q = \tfrac{a}{b}$ be represented in the base $(1, \frac{D + \sqrt{D}} {2})$ as described in \cref{impl:repr:curlO} as tuples $a_1,a_2,b_1,b_2 \in \Z$ and $q_1,q_2 \in \Q$. We can describe the equation $bq = a$ as a matrix multiplication
\[ \tilde{B} \begin{pmatrix} q_1 \\ q_2 \end{pmatrix} = \begin{pmatrix} a_1 \\ a_2 \end{pmatrix} . \]
With the inverse (if it exists), we can calculate $q$:
\[ \tilde{B}^{-1} \begin{pmatrix} a_1 \\ a_2 \end{pmatrix} = \begin{pmatrix} q_1 \\ q_2 \end{pmatrix} . \]
The multiplication formulas for $\curlO$ as described in \cref{curlOmultiplications} yield
\[ \tilde{B} = \begin{pmatrix}
b_1 & -b_2 \frac{D^2 - D}{4} \\
b_2 & b_1 + b_2 D
\end{pmatrix} . \]
Then, for the inverse, we have
\[ \tilde{B}^{-1} = \frac{1}{\det(\tilde{B})}
\begin{pmatrix}
b_1 + b_2 D & b_2 \frac{D^2 - D}{4} \\
-b_2 & b_1
\end{pmatrix} . \]
For the determinant, we have
\begin{align*}
\det(\tilde{B}) &= b_1^2 + b_1 b_2 D + b_2^2 \frac{D^2-D}{4} \\
&= (b_1 + b_2 \tfrac{D}{2})^2 \underbrace{- b_2^2 \tfrac{D^2}{4} + b_2^2 \tfrac{D^2-D}{4}}_{
= -b_2^2 \tfrac{D}{4} \ge 0} \ge 0.
\end{align*}
Note that $\det(\tilde{B}) = 0$ exactly if and only if $b_1 = b_2 = 0$, as it was expected.

This gives us some direct formulas for $q_1,q_2 \in \Q$ which are used in the \ifuncname{divmod} implementation where we select $q'_1, q'_2 \in \Z$ close to $q_1,q_2$ such that $r := a - q' b$ becomes minimal with regards to the Euclidean Norm.


\subsection{Calculating the restriction information from the map $a \rightarrow a[S]$}
\label{impl:calcMatrix}
\label{calcMatrix}

In \cref{lemma:ellipticRestriction}, we have seen that, via a matrix $S \in \PM_2(\curlO)$, we can restrict a Hermitian modula form $f : \HalfPlane_2 \rightarrow \C$ to an Elliptic modular form $f[S]$. In the whole \cref{ellipticRestriction}, we have developed the theory.

Let $N = \# \left( \invarF{\F}{\GL_2(\curlO)} \right)$ with $\invarF{\F}{\GL_2(\curlO)} = \Set{T_1,\dots,T_N}$ and let $M \in \M_{\FPrecisionLimit{S} \times N}(\Q)$ be the matrix to the linear map of Fourier expansions $a \in \Q^\F$ of Hermitian modular forms to Fourier expansions $a[S] \in \Q^{\FPrecisionLimit{S}}$ of Elliptic modular forms. In \cref{remark:how-to-calc-aS}, we have given the necessary formula
\[ M_{i,j} =  \sum_{T \in \F, \tr(S T) = i, j_T = j} \det(U_T)^{-k} \]
for the $i$-th row and $j$-th column with $0 \le i < \FPrecisionLimit{S}$ and $1 \le j \le N$, where $k$ is the weight of the Hermitian modular forms. $j_T$ is uniquely determined such that the reduced matrix of $T$ is $T_{j_T}$ and $T_{j_T}[U_T] = T$. 

In any case, we need to iterate through all of $\F$ to get all the summands in the every sum of every matrix entry. Such an iteration is described in \cref{curlFiteration}. $\F$ is quite huge (see \cref{remark:reducedCurlF}; e.g. for $D=-3$, $B=10$, we have $\# \F = 21892$) and we need to calculate the reduced matrix index $j_T$ and $U_T$ from each $T$ via \ifuncname{reduceGL} (see \cref{reduceGL}) which is heavy to calculate, thus we want to call \ifuncname{reduceGL} only once for each $T \in \F$. The calculation is still heavy, thus it was implemented in C++ for maximal performance. The algorithm works as follows:
\begin{enumerate}
\item \emph{Input}: The restriction matrix $S \in \PM_2(\curlO)$, as well as the parameters for the Hermitian modular forms, i.e. the fundamental discriminant $D$ for the underlying quadratic imaginary field, the weight $k$ and the precision limit $B_\F$.

\emph{Output}: $M \in \M_{\FPrecisionLimit{S} \times N}(\Q)$.
\item The outer loop goes through all $T \in \F$ (see \cref{curlFiteration}).
\item For each $T$, we call \ifuncname{reduceGL} (see \cref{reduceGL}) to calculate $T_{j_T}$ and $\det(U_T)$. This gives us also the matrix column $j := j_T$.
\item We also calculate $i := \tr(ST)$ to get the matrix row. In \cref{traceST}, we have shown the formula for the direct calculation and also that we have only entries in $\N_0$. We could get $i \ge \FPrecisionLimit{S}$ but we ignore those.
\item We increase the matrix entry $(i,j)$ by $\det(U_T)^{-k}$.
\end{enumerate}
This algorithm has been implemented in the function \ifuncname{calcMatrix} in the class \isymbname{ReductionMatrices_Calc} in the file \ifilename{algo_cpp.cpp}. All the state and parameters are stored in the class so that we need to copy as less data as possible for successive $S \in \PM_2(\curlO)$. That is also why the iteration through different $S \in \PM_2(\curlO)$ (see \cref{Siter}) has been done in C++.

In Python, in \ifuncname{modform_restriction_info} in the file \ifilename{algo.py}, we get an instance of that C++ class and call the function \ifuncname{calcMatrix}. This gives us the matrix $M_S$. Define its column module as
\[ \mathcal{M}_S := \Set{M_S \cdot a}{a \in \Q^\F} . \]
Via other methods in Sage, we can calculate the vector space $ \FE_{\F(S)}(\ESpace{k}{\Gamma_0(l_S)})$ of Fourier expansions of Elliptic modular forms to $\Gamma_0(l_S)$ where $l_S := det(S)$ and weight $2k$. Then consider the intersection
\[ \mathcal{M}'_S := \FE_{\F(S)}(\ESpace{2k}{\Gamma_0(l)} \cap \mathcal{M}_S . \]
Now, take them back to the Hermitian modular form space:
\[ \mathcal{M}^H_S := \Set{a \in \Q^\F}{M_S \cdot a \in \mathcal{M}'_S} . \]
In Sage, we can do that by using \ifuncname{solve_right} on the matrix $M_S$ and adding the right kernel of $M_S$.

For testing and demonstration purpose, another implementation has been done in Python in \ifuncname{calcRestrictMatrix_py} in the file \ifilename{helpers.py}. Also for testing purpose, the C++ version can be called directly via the Python function \ifuncname{calcRestrictMatrix_any}.

\subsection{Calculating the Cusp restrictions}
\label{calcMatrixTrans}
We want to develop the algorithm analogously to \cref{calcMatrix}. We have developed the neccessary basics in \cref{remark:algo-mainstep2}. We have given some $S \in \PM_2(\curlO)$ and a cusp $c \in \Q$ of $\Gamma_0(l)$ with $l = \det(S)$. Let $M_c \in \Sp_1(\Z)$ such that $M_c \infty = c$.

Recall that we have
\[ (a[S]| M_c) (p) = \overline{\det(\tilde{S})}^k \cdot \sum_{\begin{subarray}{c}
T \in \Lambda,\\
\tr\left(T \tilde{S} S \overline{\tilde{S}}^T\right) = p
\end{subarray}} a(T) \cdot e^{2 \pi i \cdot \tr\left(T \tilde{T} \overline{\tilde{S}}^T\right)} \]
for all $p \in \frac{1}{L} \N_0$ for some $L \in \N$.

We want to construct a matrix $\hat{M}_{c,S}$ such that $\hat{M}_{c,S} \cdot a = a[S]|M_c$ for some given precision limit $\F_c(S, \tilde{S})$ for the Elliptic modular forms (similar to $\FPrecisionLimit{S}$ as described in \cref{lemma:fprecisionlimit}).

Write $T = [t_1,t_2,t_4] \in \Lambda$, $S = [s_1,s_2,s_4] \in \PM_2(\curlO)$ and $\tilde{S} = \SmallMatrix{u_1}{u_2}{u_3}{u_4} \in \M_2(\K)$.  Note that $t_1,t_4,s_1,s_4 \in \Z$. Then
\begin{align*}
& \tr\left(T \tilde{S} S \overline{\tilde{S}}^T\right) \\
=\quad & {\left(t_{4} u_{4} + u_{2} \overline{t_{2}}\right)} s_{4} + {\left(t_{4} u_{3} + u_{1} \overline{t_{2}}\right)} s_{2} + {\left(t_{1} u_{2} + t_{2} u_{4}\right)} \overline{s_{2}} + {\left(t_{1} u_{1} + t_{2} u_{3}\right)} s_{1} + \overline{u_{1}} + \overline{u_{4}} \\
=\quad & {\left(s_{4} \overline{t_{2}} + t_{1} \overline{s_{2}}\right)} u_{2} + {\left(s_{4} t_{4} + t_{2} \overline{s_{2}}\right)} u_{4} + {\left(s_{1} t_{2} + s_{2} t_{4}\right)} u_{3} + {\left(s_{1} t_{1} + s_{2} \overline{t_{2}}\right)} u_{1} + \overline{u_{1}} + \overline{u_{4}} .
\end{align*}
Let $S$ and $\tilde{S}$ be fixed. Now assume $T \in \Lambda - \F_B$, i.e. $\max(t_1,t_4) \ge B$.

Case 1: $t_4 = 0$. Then we have $t_1 \ge B$ and $t_2 = 0$. And
\[ \tr\left(T \tilde{S} S \overline{\tilde{S}}^T\right) \ge
 B {s_{1}} u_{1} + {B \overline{s_{2}}} u_{2} + \overline{u_{1}} + \overline{u_{4}} \ge B(s_1 u_1 - |\overline{s_2} u_2 |) + \overline{u_1} + \overline{u_4} . \]
%
Case 2: $t_1 = 0$. Then we have $t_4 \ge B$ and $t_2 = 0$. And
\[ \tr\left(T \tilde{S} S \overline{\tilde{S}}^T\right) \ge
 B {s_4} u_4 + {B s_{2}} u_3 + \overline{u_{1}} + \overline{u_{4}} \ge B(s_4 u_4 - |s_2 u_3 |) + \overline{u_1} + \overline{u_4} . \]
%
%Case 3: $0 < t_4 < B$. Then $t_1 \ge B$.
%We also have $|t_2|^2 \le t_1 t_4 \le B t_4$, thus $X$. Then
%\[ \tr\left(T \tilde{S} S \overline{\tilde{S}}^T\right) \ge
% {\left(s_{4} \overline{t_{2}} + B \overline{s_{2}}\right)} u_{2} + {\left(s_{4} t_{4} + t_{2} \overline{s_{2}}\right)} u_{4} + {\left(s_{1} t_{2} + s_{2} t_{4}\right)} u_{3} + {\left(s_{1} t_{1} + s_{2} \overline{t_{2}}\right)} u_{1} + \overline{u_{1}} + \overline{u_{4}} . \]
Analyzing the other cases is left open for further work. We also don't prove that
\[ \tr\left(T \tilde{S} S \overline{\tilde{S}}^T\right) \in \Q_{\ge 0} \]
at this point, although the computer calculations have shown that this seems to be the case. See also \cref{remark:on-solveR-tS} for some analysis on $\tilde{S}$.

Let us assume that we have found a limit $\F_c(S,\tilde{S})$ such that $\tr\left(T \tilde{S} S \overline{\tilde{S}}^T\right) \ge \F_c(S,\tilde{S})$ for all $T \in \Lambda - \F$. Thus we can calculate the Fourier expansions of Elliptic modular forms up to precision $\F_c(S,\tilde{S})$.
%
% WRONG: (?)
%As we are dealing with vector spaces, we can ignore the constant factor $\overline{\det(\tilde{S})}^k$, because for some vector space $\mathcal{V} \subset \Q^\F$, we have $(a \mapsto a[S]| M_c) (\mathcal{V}) = (a \mapsto \frac{a[S]| M_c}{\overline{\det(\tilde{S})}^k}) (\mathcal{V})$.

For the row indices, we can use $0 \le n < L \cdot \F_c(S,\tilde{S})$ with $p = \frac{n}{L}$. The column is given by the reduced matrix index of $T$. However, the entry itself is not in $\Q$ but in the $P$-th cyclomotic field $\Q(\zeta_P)$ such that $P \cdot \tr\left(T \tilde{T} \overline{\tilde{S}}^T\right) \in \Z$ for all $T \in \Lambda$. However, there is a $(P-1)$-th dimensional basis of $\Q(\zeta_P)$ and we can use several matrices such that each represents a factor to $\zeta_P^i$ for $0 \le i < P - 1$.

To calculate the cusp expansions of the Elliptic modular forms in Sage, we use a recent work by Martin Raum which was not published yet at the time of writing. It will be published later under \cite{Raum13PSage}.


\subsection{Parallelization}
\label{impl:parallelization}

The parallelization is implemented by distributing the calculation along multiple processes. In \ifilename{utils.py}, there are the neccessary functions to use this technique.

The high level class \isymbname{Parallelization} spawns a number of independent worker processes. These are real separate processes and not forks because of issues with non-fork-safe libraries such as libBLAS. Details and references about this can be found in the source.

The communication between the worker processes and the main process is via serialization over pipes.

For our specific work, the function \ifuncname{herm_modform_space} manages a \isymbname{Parallelization} instance and delegates the calculation of the superspaces (via \ifuncname{modform_restriction_info} and \ifuncname{modform_cusp_info}) to the worker processes.

As the intersection of the superspaces takes also some time, it is also delegated to some worker processes in a non-blocking and distributing way. Details can be found in the source code.


%!TEX root =  index.tex

\section{Conclusion and further work}

We have developed and implemented an generic algorithm to calculate the vector space of Fourier expansions of Hermitian modular forms.

We haven't gotten any results yet as it was also noted in \cref{chapter:impl}. Many parts of the code have been tested in various ways but as we don't have any results yet, we cannot tell whether it is all correct and it needs more testing.

Also the precision limit $\FPrecisionLimit{S}$ (see \cref{lemma:fprecisionlimit}) could be improved so that the information gain is bigger for more $S \in \PM_2(\curlO)$. Right now, only a few $S$ are usable for us.

A few more details in \cref{calcMatrixTrans} about the cusp restriction information gain need to be worked out, such as a good precision limit $\F_c(S,\tilde{S})$.


% http://tex.stackexchange.com/questions/8458/making-the-bibliography-appear-in-the-table-of-contents
% http://www.automata.rwth-aachen.de/material/skripte/latex/latex.pdf
% 9.2ff
\bibliographystyle{alpha}
\bibliography{bibliography}

\printindex

\end{document}
