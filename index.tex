\documentclass[twoside,openright]{article}
%%!TEX encoding = UTF-8 Unicode
\usepackage[utf8]{inputenc}
\usepackage{palatino} %Schriftart
%\usepackage{ngerman}
%\usepackage[ps2pdf,a4paper,colorlinks]{hyperref}
%\usepackage[a4paper,colorlinks]{hyperref}
\usepackage[a4paper]{hyperref}
\usepackage[a4paper,%
	inner=3.5cm,%
	outer=3.5cm,%
	top=4cm,%
	bottom=4cm,%
	marginparwidth=2.5cm,%
	marginparsep=0.3cm,%
	includehead]{geometry}
\usepackage{makeidx}
\usepackage[nottoc,numbib]{tocbibind}
\usepackage{titlesec}
\usepackage[fleqn]{amsmath}
\usepackage{amsthm}
\usepackage{amstext}
\usepackage{amssymb}
\usepackage{mathtools}
\usepackage{xparse}
\usepackage{url}
\usepackage{cleveref}
\usepackage{mdframed}
\usepackage{tikz}
\usetikzlibrary{automata,positioning}
% doesnt work?
%\usepackage{vaucanson-g}
% evtl auch gastex. siehe: http://www.automata.rwth-aachen.de/material/skripte/latex/latex.pdf

\pagestyle{headings}

\hypersetup{%
	pdftitle = {Language Operations and a Structure Theory of ω-Languages},%
	pdfsubject = {},%
	pdfauthor = {Albert Zeyer}%
}

% funktioniert nicht?
%\makeidx

%!TEX root =  index.tex

% some useful stuff:
% http://www.automata.rwth-aachen.de/material/skripte/latex/latex.pdf
% http://en.wikibooks.org/wiki/LaTeX/Mathematics
% http://en.wikibooks.org/wiki/LaTeX/Advanced_Mathematics
% http://en.wikibooks.org/wiki/LaTeX/Theorems

\theoremstyle{plain}\newtheorem{lemma}{Lemma}[section]
\theoremstyle{plain}\newtheorem{theorem}[lemma]{Theorem}
\theoremstyle{definition}\newtheorem{mydef}[lemma]{Definition}
\theoremstyle{definition}\newtheorem{algo}[lemma]{Algorithm}
\theoremstyle{plain}\newtheorem{example}[lemma]{Example}
\theoremstyle{definition}\newtheorem{sexample}[lemma]{Example}

% http://tex.stackexchange.com/questions/5767/how-to-get-more-complete-references
\Crefname{section}{Chapter}{Chapters}
\crefname{section}{chapter}{chapters}
\Crefname{subsection}{Section}{Sections}
\crefname{subsection}{section}{sections}

\crefname{sexample}{example}{examples}
\crefname{mydef}{definition}{definitions}

\newenvironment{simpleexample}[0]%
{\begin{sexample}}%
{\qed \end{sexample}}

\newcommand{\F}{\mathcal{F}}
\newcommand{\K}{\mathcal{K}}
\newcommand{\Z}{\mathbb{Z}}
\newcommand{\Q}{\mathbb{Q}}
\newcommand{\C}{\mathbb{C}}
\newcommand{\R}{\mathbb{R}}
\newcommand{\N}{\mathbb{N}}
\newcommand{\B}{\mathbb{B}}
\newcommand{\HalfPlane}{\mathbb{H}}
\newcommand{\Power}{\mathcal{P}}

\newcommand{\mathtext}[1]{\textup{\textrm{#1}}}
\newcommand{\PT}{\mathtext{PT}}
\newcommand{\posPT}{\mathtext{pos-PT}}
\newcommand{\LT}{\mathtext{LT}}
\newcommand{\LTT}{\mathtext{LTT}}

% got some help here: http://tex.stackexchange.com/questions/13554/define-something-like-lim-but-for-another-name

\newcommand{\M}{\operatorname{M}}
\newcommand{\GL}{\operatorname{GL}}
\newcommand{\SL}{\operatorname{SL}}
\newcommand{\Sp}{\operatorname{Sp}}
\newcommand{\Orth}{\operatorname{O}}
%\newcommand{\det}{\operatorname{det}}

\newcommand{\SmallMatrix}[4]{\left( \begin{array}{cc}
#1 & #2 \\
#3 & #4 \end{array} \right)}

\newcommand{\existsinf}{\exists^\omega}
\newcommand{\overx}{\overset{\times}}
%\newcommand{\overx}{\stackrel{\times}}
\newcommand{\Ax}{\overx{\A}}

\newcommand{\defword}[1]{{\bf #1}}

% inspired by http://ftp.fernuni-hagen.de/ftp-dir/pub/mirrors/www.ctan.org/macros/latex/contrib/braket/braket.sty
\def\mid@vertical{\mskip1mu\vrule\mskip1mu}
\def\midvert{\egroup\;\mid@vertical\;\bgroup}
\NewDocumentCommand\Set{mg}{%
    \IfNoValueTF{#2}{%
        \ensuremath{\left\{ #1 \right\}}%
    }{%
        \ensuremath{\left\{ {#1} \;\mid@vertical\; {#2} \right\}}%
    }%
}

%\newcommand{\SetS}[1]{\bigl\{ #1 \bigr\}}
%\newcommand{\SetC}[2]{\bigl\{ #1 \bigm| #2 \bigr\}}
%\DeclarePairedDelimiterX\SetC[2]{\lbrace}{\rbrace}{ #1 \,\delimsize|\, #2 }

%\newcommand{\abs}[1]{\mathopen| #1 \mathclose|}
%\newcommand{\Abs}[1]{\left| #1 \right|}
\newcommand{\abs}[1]{\left| #1 \right|}

% http://de.wikibooks.org/wiki/LaTeX-W%C3%B6rterbuch:_today
\def\monthgerman{\ifcase\month \or
  Januar\or Februar\or M\"arz\or April\or Mai\or Juni\or
  Juli\or August\or September\or Oktober\or November\or Dezember\fi}
\def\todaygerman{\number\day.~\monthgerman\space\number\year}


% new page for every section
\let\stdsection\section
\renewcommand\section{\newpage\stdsection}


\begin{document}
\title{Language Operations and a Structure Theory of $\omega$-Languages}
\author{Albert Zeyer}
\date{\today}

% http://en.wikibooks.org/wiki/LaTeX/Title_Creation
\begin{titlepage}
\begin{center}
\setlength{\parskip}{2ex plus0.5ex minus0.2ex}
\setlength{\baselineskip}{5ex}
\textsc{\LARGE Language Operations and a Structure Theory of $\omega$-Languages}\\[1.5cm]

\setlength{\baselineskip}{3ex}

\textsc{Diploma Thesis} \\
in Computer Science \\[0.7cm]

by \\
Albert Zeyer \\[3cm]

submitted to the \\
Faculty of Mathematics, Computer Science and Natural Science of \\
RWTH Aachen University \\[1.5cm]

July 2012 \\
revised version from \today \\[1.5cm]

Supervisor: Prof. Dr. Dr.h.c. Wolfgang Thomas \\
Second examiner: PD Dr. Christof Löding \\[1.5cm]

written at the \\
Chair of Computer Science 7 \\
Logic and Theory of Discrete Systems \\
Prof. Dr. Dr.h.c. Wolfgang Thomas

\end{center}
\end{titlepage}


% dont need that for the revised version
%% empty page
%\newpage
%\thispagestyle{empty}
%\mbox{}
%
%\begin{titlepage}
%\setlength{\parindent}{0pt}
%\setlength{\parskip}{3ex}
%\textbf{\large Erklärung}
%
%Hiermit versichere ich, dass ich diese Arbeit selbstständig verfasst und keine anderen als die angegebenen Quellen und Hilfsmittel benutzt sowie Zitate kenntlich gemacht habe.\\
%
%Aachen, den \todaygerman
%\end{titlepage}
%
%% empty page
%\newpage
%\thispagestyle{empty}
%\mbox{}

\newpage
\thispagestyle{empty}
\setlength{\parskip}{0.7ex}
\tableofcontents
\newpage


% http://www.automata.rwth-aachen.de/material/skripte/latex/latex.pdf
\setlength{\parindent}{0pt}
\setlength{\parskip}{2ex plus0.5ex minus0.2ex}
\setlength{\baselineskip}{3ex}
\renewcommand{\baselinestretch}{1.5}

% http://www.ctex.org/documents/packages/layout/titlesec.pdf
\titleformat{\section}[display]{\LARGE\bfseries}{Chapter \thesection}{0.5em}{}{}
\titlespacing*{\section}{0pt}{0pt}{2em}

%!TEX root =  index.tex

\section{Introduction}

In \cite{PoorYuen07Comp}, spaces of Siegel modular cusp forms are calculated.

We are doing the same for hermitian modular forms.





\input{reg-omega-lang}
\input{star-omega-ops}
\input{generic-star-omega-results}
\input{star-languages}
%!TEX root =  index.tex

\section{Conclusion}

Blub


% http://tex.stackexchange.com/questions/8458/making-the-bibliography-appear-in-the-table-of-contents
% http://www.automata.rwth-aachen.de/material/skripte/latex/latex.pdf
% 9.2ff
\bibliographystyle{alpha}
\bibliography{bibliography}

\printindex

\end{document}
