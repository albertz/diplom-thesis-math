%!TEX root =  index.tex

\section{Background results}

\subsection{Preliminaries}

$\N$ denotes the set $\Set{1,2,3,\dots}$, $\N_0 = \N \cup \Set{0}$ and $\Z$ are all integers. $\Q$ are all the rational numbers, $\R$ are the real numbers and $\C$ are the complex numbers.

Let $\M_n(R)$ be the set of all $n \times n$ matrices over some commutative ring $R$.
Likewise, $\M_n^T(R)$ are the symetric $n \times n$ matrices.
A matrix $Y \in \M_n(\R)$ is greater $0$ iff $\forall x \in \R^n - \Set{0} \colon Y[x] := x^T Y x > 0$.
Let $\HalfPlane_n := \Set{Z = X + iY \in \M_n^T(\C)}{Y > 0}$.
Thus, $\HalfPlane_1$ is the \defword{Poincaré upper half plane}.

The \defword{general linear group} is defined by $\GL_n(R) = \Set{X \in \M_n(R)}{\text{$\det(X)$ is a unit in $R$}}$
and the \defword{special linear group} by $\SL_n(R) = \Set{X \in \M_n(R)}{\det(X) = 1}$. % Poor
The \defword{orthogonal group} is defined by $\Orth_n(R) = \Set{X \in \GL_n(R)}{X^T 1_n X = 1_n}$. %Poor
The \defword{symplectic group} is defined by $\Sp_n(R) = \Set{X \in \GL_{2n}(R)}{X^T J_n X = J_n}$ %Poor
% Tippfehler in Poor: Dort sind es n*n Matrixen, es müssen aber 2n*2n Matrixen sein.
where $J_n := \SmallMatrix{0}{1_n}{-1_n}{0} \in \SL_{2n}(R)$. % Poor
% In Dern ist U_2(\curlO) = \Sp_2(\curlO).
% Dern nennt das, und wir auch: Hermitian modular group
% U steht für unitäre Gruppe.
% Sp steht für symplectische Gruppe.
$\Sp_n(R)$ will also be called the \defword{Hermitian modular group}.
% Weil-Darstellung ist eine gewisse Darstellung von symplektischen Gruppen. Aber irrelevant für mich.
% In dem Zusammenhang waren auch vektorwertigen Modulformen, die auch nicht relevant für mich sind.

% Groß/Kleinschreibung:
% Hermitian modular form
% Siegel modular cusp form
% In Überschriften fast alles (alle Nomen) kapitalisiert

% Skript Krieg, p.49. Aber die folgende Def ist woanders her, glaub ich...?
% cusp = Spitzenform
A \defword{Siegel modular cusp form} of degree $n\in\N$ for some $\Gamma \subseteq \Sp_n(\Z)$ is a holomorphic function
\[ f \colon \HalfPlane_n \rightarrow \C \]
with
\begin{align*}
(1) \ \ & f |_k y = f \ \ \forall \ y \in \Gamma \\
(2) \ \ & \text{for } n = 1 \colon f(Z) = O(1) \ \text{ for } Z \rightarrow i \infty
\end{align*}
where
\[ \left( f|_k \SimpleMatrix{A}{B}{C}{D} \right) (Z) =
f((AZ + B)(CZ + D)^{-1}) \cdot \det(CZ + D)^{-k} \]
%with $Z = S \tau$. https://mail.google.com/mail/u/0/#label/Diplomarbeit/13b471d95eb713e9
with $Z \in \HalfPlane_n$, $\SmallMatrix{A}{B}{C}{D} \in \Gamma$.

% Def 1.8 bei Dern
% Bei Dern: j((A B C D),Z) = det(CZ + D)
% Multiplikatorsystem -> multiplier
A \defword{Hermitian modular form} of degree $n\in\N$
is a holomorphic function
\[ f \colon \HalfPlane_n \rightarrow \C \]
with weight $k\in \Z$ for some $\Gamma \subseteq \Sp_n(\curlO)$, $\curlO \subseteq \Q(\sqrt{-\Delta})$, $\Delta \in \N$, $\nu \colon \Gamma \rightarrow \C^\times$ is an abel character of $\Sp_n(\curlO)$, with
\begin{align*}
(1) \ \ & f(M \cdot Z) = \nu(M) \det(CZ + D)^k f(Z), \ \ \ \ M = \SmallMatrix{*}{*}{C}{D} \in \Gamma, Z \in \HalfPlane_n , \\
(2) \ \ & \text{for } n = 1 \colon \ \text{$f$ is holomorphic in all peaks} .
\end{align*}

$[\Gamma, k, \nu]$ denotes the vector space of such hermitian modular forms.

In this work, we will concentrate on Hermitian Modular forms of degree 2. We will start with $\Delta \in \Set{3,4,8}$.
% weight k is fixed in the algorithm, but the algorithm allows any k.

Note that if $\Delta$ is fundamental, we have
\begin{align*}
\curlO = &\ \Z +  \Z \frac{-\Delta+\sqrt{-\Delta}}{2} , \\
\curlO^\# = & \ \Z \frac{i}{\sqrt{-\Delta}} + \Z \frac{1 + \sqrt{-\Delta}}{2} .
\end{align*}

From now on, we will always work with Hermitian modular forms of degree 2, i.e. we will always have $n=2$, except if otherwise stated.

