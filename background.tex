%!TEX root =  index.tex

\section{Background results}

\subsection{Preliminaries}

Let $\M_n(\K)$ be the set of all $n \times n$ matrices over some field $\K$.
Likewise, $\M_n^T(\K)$ are the symetric $n \times n$ matrices.
A matrix $Y \in \M_n(\R)$ is greater $0$ iff $\forall x \in \R^n - \Set{0} \colon Y[x] := x^T Y x > 0$.
Let $\HalfPlane_n := \Set{Z = X + iY \in \M_n^T(\C)}{Y > 0}$.
Thus, $\HalfPlane_1$ is the Poincaré upper half plane.

The general linear group is defined by $\GL_n(\K) = \Set{X \in \M_n(\K)}{\det(X) \neq 0}$
and the special linear group by $\SL_n(\K) = \Set{X \in \M_n(\K)}{\det(X) = 1}$. % Poor
The orthogonal group is defined by $\Orth_n(\K) = \Set{X \in \GL_n(\K)}{X^T 1_n X = 1_n}$. %Poor
The symplectic group is defined by $\Sp_n(\K) = \Set{X \in \GL_n(\K)}{X^T J_n X = J_n}$ %Poor
where $J_n := \SmallMatrix{0}{1_n}{-1_n}{0} \in \SL_{2n}(\K)$. % Poor

A \defword{Siegel Modular Cusp form} of degree $n\in\N$ for some $\Gamma \subseteq \Sp_n(\Z)$ is a holomorphic function
\[ f \colon \HalfPlane_n \rightarrow \C \]
with
\begin{align*}
(1) \ \ & f |_k y = f \ \ \forall \ y \in \Gamma \\
(2) \ \ & \text{for } n = 1 \colon f(Z) = O(1) \ \text{ for } Z \rightarrow i \infty
\end{align*}
where
\[ \left( f|_k \SmallMatrix{A}{B}{C}{D} \right) (Z) =
f((AZ + B)(CZ + D)^{-1}) \cdot \det(CZ + D)^{-k} \]
%with $Z = S \tau$. https://mail.google.com/mail/u/0/#label/Diplomarbeit/13b471d95eb713e9
with $Z \in \HalfPlane_n$, $\SmallMatrix{A}{B}{C}{D} \in \Gamma$.

% Def 1.8 bei Dern
% Bei Dern: j((A B C D),Z) = det(CZ + D)
% Multiplikatorsystem -> multiplier
A \defword{Hermitian Modular form} of degree $n\in\N$
is a holomorphic function
\[ f \colon \HalfPlane_n \rightarrow \C \]
with weight $k\in \Z$ for some $\Gamma \subseteq \Sp_n(\curlO)$, $\curlO \subseteq \Q(\sqrt{D})$, $D \in -\N$, $\nu \colon \Gamma \rightarrow \C^\times$ is an abel character of $\Sp_n(\curlO)$, with
\begin{align*}
(1) \ \ & f(M \cdot Z) = \nu(M) \det(CZ + D)^k f(Z), \ \ \ \ M = \SmallMatrix{*}{*}{C}{D} \in \Gamma, Z \in \HalfPlane_n , \\
(2) \ \ & \text{for } n = 1 \colon \ \text{$f$ is holomorphic in all peaks} .
\end{align*}


In this work, we will concentrate on Hermitian Modular forms of degree 2.
